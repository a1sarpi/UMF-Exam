\subsection{Принцип максимального значения для параболического уравнения и теорема единственности смешанной краевой задачи в ограниченной области}
\label{maximum_principle}

\begin{theorem}[о максимальном значении] Функция $ u(x, t) $, которая на промежутках $ 0 < x < l $, $ 0 <
  t \leqslant T$ удовлетворяет уравнению
\begin{equation}
  \label{eq:9/teplo_eq}
  u_t = a^2u_{xx},
\end{equation}
определённая и непрерывная внутри соответствующей замкнутой области, принимает максимальное и минимальное значения либо в начальный момент, либо в
точке границы $ x = 0 $ или $ x = l $.\end{theorem}

Физический смысл этой теоремы очевиден: если
температура на
границе и в начальный момент не превосходит некоторого значения
$ M $, то при отсутствии источников внутри тела не может создаваться
температура, б\'{о}льшая $ M $. Остановимся сначала на доказательстве
теоремы для максимального значения.

\begin{proof}Пойдём от противного. Обозначим через $ M $ максимум функции в указанных трёх
множествах: $ t = 0 $, $ x = 0 $ и $ x = l $. Предположим, что есть такая
точка\footnote{Такая точка должна существовать, поскольку функция непрерывна на компакте.}
$ (x_0, t_0) $, не принадлежащая перечисленным множествам, где $ u(x, t) $
достигает своего максимума, то есть 
\[
    u(x_0, t_0) = M + \varepsilon, \quad \varepsilon > 0.
\]
Чтобы прийти к противоречию, найдём такую точку $ (x_1, t_1) $, в которой
(вопреки уравнению \eqref{eq:9/teplo_eq}) $ u_{xx} \leqslant 0 $, $ u_t > 0 $.
Воспользуемся вспомогательной функцией  
\[
    v(x, t) := u(x, t) + k(t_0 - t).
\]
Выберем $ 0 < k < \varepsilon/(2T) $. Тогда во всех трёх множествах $
v(x, t) \leqslant M + \varepsilon/2 $. Тогда максимум функции $ v $ достигается
где-то ещё, поскольку хотя бы $ v(x_0, t_0) = M + \varepsilon $, --- в точке,
которую назовём $ (x_1, t_1) $.
Перечислим некоторые необходимые условия этого события\footnote{Действительно,
  будь вторая производная строго больше нуля, то получили бы локальный минимум функции; производная по времени, конечно, может быть больше нуля только в точке $ T $.}: 
\begin{gather*}
    \frac{\partial v}{\partial x}(x_1, t_1) = 0, \quad \frac{\partial^2
    v}{\partial x^2}(x_1, t_1) = \frac{\partial^2 u}{\partial x^2}(x_1, t_1) \leqslant 0,\\
    \frac{\partial v}{\partial t}(x_1, t_1) = \frac{\partial u}{\partial t}(x_1,
    t_1) - k \geqslant 0,
\end{gather*}
откуда и следует доказываемое.
\end{proof}
Чтобы доказать аналогичную теорему для минимального значения, достаточно перейти
к функции $ -u $.

\begin{theorem}[единственности]
  Если две функции $ u_1 $ и $ u_2 $, определённые и непрерывные в области $
  0\leqslant x\leqslant l $, $ 0 \leqslant t \leqslant T $, удовлетворяют
  уравнению теплопроводности  
  \[
    u_t = a^2 u_{xx} + f(x, t), \quad 0 < x < l, \ 0 < t \leqslant T,
  \]
  одинаковым начальным и граничным условиям  
  \begin{align*}
    u_1(x, 0) &= u_2(x, 0) = \varphi(x),\\
    u_1(0, t) &= u_2(0, t) = \mu_1(t),\\
    u_1(l, t) &= u_2(l, t) = \mu_2(t),
  \end{align*}
  то $ u_1 = u_2 $.
\end{theorem}

\begin{proof}
  Рассмотрим функцию $ v = u_2 - u_1 $, которая, конечно, является решением
  соответствующего однородного уравнения. Применяя принцип максимального
  значения к этой функции, приходим к завершению доказательства.
\end{proof}
  
