\subsection{Краевые задачи для уравнений Лапласа и Пуассона в шаровом слое.}

Вот такое уравнение называется первой краевой задачей для уравнения Пуассона в шаровом слое:
\[
  \begin{cases}
    \Delta u = f(x, y, z), (x, y, z) \in \Omega, \\
    \left. u \right|_{r = a} = g(\varphi, \theta), \\
    \left. u \right|_{r = b} = h(\varphi, \theta),
  \end{cases}
\]
где $\Omega = \left\{ (r, \theta, \varphi) : a < r < b \right\} $.
При $f(x, y, z) \equiv 0$ уравнение называется Лапласом.

Если на границах заданы производные вдоль внешней нормали, то это вторая краевая задача,
а если задана линейная комбинация, то это третья краевая задача.

\paragraph{Способы решения}
\begin{itemize}
  \item Найти частное решение $w$ для неоднородного уравнения, то есть найти какое-нибудь 
    решение задачи $\Delta w = f(x, y, z)$. Чаще всего такое можно сделать, если правая часть 
    представляется в каком-то специальном виде, например,
    если $f(x, y, z) = r^2 \cos 2\varphi$, то $w$ легко ищется в виде $A r^\alpha \cos 2\varphi$,
    если $f(x, y, z) = \operatorname{const}$, то $w = A r^\alpha$.
    Тогда останется найти только часть решения $v$, удовлетворяющую уравнению:
    \[
      \begin{cases}
        \Delta v = 0, \\
        \left. v \right|_{r=a} = g(\varphi, \theta) - \left. w \right|_{r=a}, \\
        \left. v \right|_{r=b} = h(\varphi, \theta) - \left. w \right|_{r=b}.
      \end{cases}
    \]

  \item Метод функций Грина.
    % TODO про функции Грина не разбираюсь

  \item Метод разделения переменных, о нём ниже.
\end{itemize}

\paragraph{Разделение переменных}

Если представить решение в виде: $u = R(r) W(\varhpi, \theta)$, то
\begin{multline*}
  \left( r^2 R' \right)' W
  + \dfrac{1}{\sin \theta} R \dfrac{\partial }{\partial \theta} \left( \sin\theta \dfrac{\partial W}{\partial \theta} \right) 
  + \dfrac{1}{\sin^2 \theta} R \dfrac{\partial^2 W}{\partial \varphi^2} = 0
  \Rightarrow \\
  \Rightarrow
  \dfrac{(r^2 R')'}{R}
  = - \dfrac
    {\dfrac{1}{\sin \theta} \dfrac{\partial }{\partial \theta} \left( \sin\theta \dfrac{\partial W}{\partial \theta} \right) + \dfrac{1}{\sin^2 \theta} \dfrac{\partial^2 W}{\partial \varphi^2}}
    {W} = - \lambda
\end{multline*}
далее разделим $W = \Theta(\theta) \Phi(\varphi)$:
\[
  \sin\theta \Phi (\sin\theta \Theta')' + \lambda \sin^2\theta \Theta\Phi
  + \Theta \Phi'' = 0
  \Rightarrow
  - \dfrac{\sin\theta (\sin\theta \Theta')' + \lambda \sin^2\theta \Theta}{\Theta}
  = \dfrac{\Phi''}{\Phi} = -\mu,
\]
решением задачи для $\Phi(\varphi)$ с дополнительным условием однозначности является:
\[
  \Phi_n (\varphi) = a_n \cos(n \varphi) + b_n \sin(n \varphi), \mu = n^2,
  n \in \mathbb{N} \cup \{0\}
\]
Тогда задача для $\Theta$ принимает вид:
\[
  \begin{cases}  
    \sin\theta (\sin\theta \Theta')' + \lambda \sin^2\theta \Theta - n^2 \Theta = 0, \\
    |\Theta(0)| < \infty, |\Theta (\pi)| < \infty.
  \end{cases}
\]
Решением такой задачи являются полиномы Лежандра (это надо проверить):
\[
  \Theta_{nm} (\theta) = P_{m}^{(n)} (\cos\theta),
  \lambda_m = m(m+1), m \in \left\{ 0, 1, \dots, n \right\} 
\]

Теперь можно вернуться к неоднородной задаче, так как мы получили ортогональную систему.
