\subsection{Задача без начального условия для уравнения теплопроводности}
Мотивацию см. в разделе \ref{sec:teplo} (предельные случаи). 
\paragraph{Полубесконечный стержень.} Рассмотрим первую
краевую задачу для полубесконечного стержня без начальных условий. Будем
считать, что, во-первых, функция $ u $ ограничена, а во-вторых,  
\begin{equation}
  \label{eq:12/main}
  \begin{cases}
    u_t = a^2u_{xx}, & x>0,\\
    u(0, t) = A\cos\omega t.\\
  \end{cases}
\end{equation}
Основное уравнение можно переписать в виде $ Lu = 0  $. Имея некоторое комплексное
решение этого уравнения $ z $, можно утверждать, что решениями также будут функции $ \Re(z) $
и $ \Im(z) $. Причём в случае, если $ z $ удовлетворяет условию $ z(0, t) =
Ae^{i\omega t} $, можно сказать даже, что $ \Re(z) $ будет решением всей системы
\eqref{eq:12/main} (см. \emph{формулу Эйлера}). Так мы переходим к соответствующей комплексной задаче.

Будем искать её решение в виде\footnote{См. предыдущие тепловые задачи, где после
разделения переменных всегда получалось что-то подобное (?).} $ z(x, t) = Ae^{\alpha x + \beta t} $, где после
подстановки находим $ \alpha^2 = \beta/a^2 $, $ \beta = i\omega $. Учитывая
тождество $ \sqrt i = (1+i)/\sqrt 2 $, можем заключить теперь, что 
\begin{align*}
  z(x, t) &= A\exp \left[ \pm \sqrt{ \frac{\omega}{2a^2}} x + i \left( \pm
  \sqrt{\frac{\omega}{2a^2}}x + \omega t \right)  \right],\\
      u(x, t) &= \Re(z) =  A\exp \left( \pm \sqrt{ \frac{\omega}{2a^2}} x\right)
      \cos \left( \pm
  \sqrt{\frac{\omega}{2a^2}}x + \omega t \right),
\end{align*}
где, конечно, стоит выбрать знак <<минус>>, поскольку по условию решение должно
быть ограниченным.

\paragraph{Ограниченный стержень.}
Аналогично решается задача  
\begin{equation}
  \label{eq:12/main2}
  \begin{cases}
    u_t = a^2 u_{xx}, & 0 < x < l,\\
    u(0, t) = A\cos\omega t,\\
    u(l, t) = 0.
  \end{cases}
\end{equation}
Перепишем граничные условия в виде $ \hat u(0, t) = Ae^{-i\omega t} $, $ \hat
u(l, t) = 0 $, а решение будем искать в форме\footnote{(?)} $ \hat u(x, t) =
X(x)e^{-\omega t} $. После подстановки получаем задачу
\begin{gather*}
  X'' + \gamma^2X =0,\qquad \text{где }\gamma = \sqrt{
  \frac{\omega}{2a^2}}(1+i),\\
  X(0) = A,\quad
  X(l) = 0.
\end{gather*}
Отсюда (?) 
\[
    X(x) = A \frac{\sin\gamma(l-x)}{\sin\gamma l}
\]
и решение уравнения \eqref{eq:12/main2} есть функция 
\[
  u(x, t) = \Re(X)\cos\omega t + \Im(X)\sin\omega t.
\]
Найти $ \Re(X) $, $ \Im(X) $ можно после больших выкладок с помощью формул $
\sin(z) = (e^{iz} - e^{-iz})/(2i) $, $ \cos(z)= (e^{iz} - e^{-iz})/2$, а также
формулы для синуса суммы.



Если граничная функция представляет собой комбинацию 
гармоник разных частот, то решение такой задачи может быть получено как
суперпозиция решений, соответствующих отдельным гармоникам.
