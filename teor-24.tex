\subsection{Полиномы Лежандра¸их свойства. Формула Родриго. Рекуррентные соотношения. Задача Штурма – Лиувилля на сфере. Присоединенные функции Лежандра.}

%Самарский стр. 709
%aut: Денис


\textbf{Полиномы Лежандра, формула Родриго}

Полиномы Лежандра тесно связаны с ФР уравнения Лапласа $\frac{1}{R}$, где $R$ - расстояние от $М$ до $M_0$. Пусть $r, r_0$ - радиус-векторы этих точек, а $\theta$ - угол между ними

\[
\frac{1}{R}=\frac{1}{\sqrt{r_{0}^{2}+r^{2}-2 r r_{0} \cos \theta}}=\begin{cases}
	\frac{1}{r_{0}} \frac{1}{\sqrt{1+\rho^{2}-2 \rho x}} & \text { для } \quad r<r_{0}, \\
	\frac{1}{r} \frac{1}{\sqrt{1+\rho^{2}-2 \rho x}} & \text { для } \quad r>r_{0},
\end{cases}.
\]

где $x=\cos \theta(-1 \leq x \leq 1)$ и $\rho=r / r_{0}<1$ или $\rho=r_{0} / r<1$ (в обоих случаях $\rho$ меньше единицы).

Функция
\[
\Psi(\rho, x)=\frac{1}{\sqrt{1+\rho^{2}-2 \rho x}} \quad(0<\rho<1, \quad-1 \leq x \leq 1)
\]
называется производящей функцией полиномов Лежандра.

Разложим функцию $\Psi(\rho, x)$ в ряд по степеням $\rho$ :
\[
\Psi(\rho, x)=\sum_{n=0}^{\infty} P_{n}(x) \rho^{n} .
\]

Коэффициенты $P_{n}(x)$ разложения формулы являются полиномами $n$-й степени и называются полиномами Лежандра.
В силу теоремы Коши из формулы следует, что
\[
P_{n}(x)=\left.\frac{1}{n !} \frac{\partial^{n} \Psi}{\partial \rho^{n}}\right|_{\rho=0}=\frac{1}{2 \pi i} \int_{C} \frac{\Psi(\zeta, x)}{\zeta^{n+1}} d \zeta,
\]
где $C$ - любой замкнутый контур в плоскости комплексного переменного $\zeta=\xi+i \eta$, содержащей точку $\zeta=0$. Полагая $\sqrt{1-2 x \zeta+\zeta^{2}}=$ $=1-\zeta z$, находим $\zeta=2(z-x) /\left(z^{2}-1\right), d \zeta=2(1-\zeta z) d z /\left(z^{2}-1\right)$, $\Psi(\zeta, x) d \zeta=2 d z /\left(z^{2}-1\right)$.

Формула выше примет вид
\[
P_{n}(x)=\frac{1}{2^{n+1} \pi i} \int_{C_{1}} \frac{\left(z^{2}-1\right)^{n}}{(z-x)^{n+1}} d z
\]
где $C_1$ - любой контур окружающий точку $z = x$.

Учитывая, что
\[
\frac{1}{2 \pi i} \int_{C_{1}} \frac{\left(z^{2}-1\right)^{n}}{z-x} d z=\left(x^{2}-1\right)^{n},
\]

и пользуясь формулой для производной
\[
\frac{d^{n}}{d x^{n}} \int_{C_{1}} \frac{\left(z^{2}-1\right)^{n}}{z-x} d z=n ! \int_{C_{1}} \frac{\left(z^{2}-1\right)^{n}}{(z-x)^{n+1}} d z,
\]

получаем формулу для $P_{n}(x)$ :
\[
P_{n}(x)=\frac{1}{2^{n} n !} \frac{d^{n}}{d x^{n}}\left[\left(x^{2}-1\right)^{n}\right] .
\]

Формула называется дифференциальной формулой для полиномов Лежандра или формулой Родрига.

Из данной формулы непосредственно видно, что: 1) $P_{n}(x)$ есть полином степени $n ; 2) P_{n}(x)$ содержит только степени $x$ той четности, что и номер $n$, так что
\[
P_{n}(-x)=(-1)^{n} P_{n}(x) .
\]

Полагая $x=1$, находим
\[
\Psi(\rho, 1)=\frac{1}{1-\rho}=1+\rho+\ldots+\rho^{n}+\ldots=\sum_{n=0}^{\infty} P_{n}(1) \rho^{n},
\]
т. е. $P_{n}(1)=1$, и в силу $(7)$
\[
P_{n}(-1)=(-1)^{n} .
\]


\textbf{Рекуррентные формулы. }

Дифференцируя $\Psi(\rho, x)$ по $\rho$ и $x$, получаем два тождества:
\[
\begin{aligned}
	\left(1-2 \rho x+\rho^{2}\right) \Psi_{\rho}-(x-\rho) \Psi=0, \\
	\left(1-2 \rho x+\rho^{2}\right) \Psi_{x}-\rho \Psi=0 .
\end{aligned}
\]

Запишем в виде степенного ряда относительно $\rho$, подставив в нее ряд $\Psi(\rho, x)=\sum_{n=0}^{\infty} P_{n}(x) \rho^{n}$ для $\Psi$ и ряд $\Psi_{\rho}=\sum_{n=0}^{\infty}(n+1) P_{n+1}(x) \rho^{n}$.
Коэффициент при $\rho^{n}$ полученного ряда равен нулю при всех $x$ :
\[
(n+1) P_{n+1}(x)-x(2 n+1) P_{n}(x)+n P_{n-1}(x)=0 .
\]

Это тождество есть рекуррентная формула, связывающая три последовательных полинома. Она позволяет найти последовательно все $P_{n}(x)$ $(n>1)$, если учесть, что
\[
P_{0}(x)=1, \quad P_{1}(x)=x .
\]

Так, например, полагая $n=1$, находим $P_{2}(x)=1 / 2 \cdot\left(3 x^{2}-1\right)$.
Выведем еще две рекуррентные формулы:
\[
n P_{n}(x)-x P_{n}^{\prime}(x)+P_{n-1}^{\prime}(x)=0,
\]

или
\[
\begin{array}{c}
	P_{n-1}^{\prime}(x)=x P_{n}^{\prime}(x)-n P_{n}(x), \\
	P_{n}^{\prime}(x)-x P_{n-1}^{\prime}(x)-n P_{n-1}(x)=0 .
\end{array}
\]

\textbf{Присоединенные функции Лежандра}

Рассмотрим следующую задачу. Найти собственные значения и собственные функции уравнения
\[
\frac{d}{d x}\left[\left(1-x^{2}\right) \frac{d y}{d x}\right]+\left(\lambda-\frac{m^{2}}{1-x^{2}}\right) y=0, \quad-1<x<1,
\]

при условии ограниченности
\[
|y( \pm 1)|<\infty .
\]

Решение уравнение естественно искать в виде
\[
y(x)=\left(1-x^{2}\right)^{m / 2} v(x), \quad v( \pm 1) \neq 0 .
\]

Подставив в исходное, найдем
\[
\left(1-x^{2}\right) v^{\prime \prime}-2(m+1) v^{\prime}+[\lambda-m(m+1)] v=0 .
\]

Нетривиальное ограниченное решение $z=P_{n}(x)$ уравнения Лежандра существует лишь при $\lambda=n(n+1)$, где $n$ - целое положительное число. Отсюда следует, что
\[
v(x)=\frac{d^{m} P_{n}}{d x^{m}}, \quad \lambda=n(n+1)
\]

есть решение уравнения, а функция
\[
P_{n}^{(m)}(x)=\left(1-x^{2}\right)^{m / 2} \frac{d^{m} P_{n}}{d x^{m}}
\]
есть собственная функция задачи, соответствующая собственному значению
\[
\lambda_{n}=n(n+1), \quad n=1,2, \ldots
\]

Функция $P_{n}^{(m)}(x)$ называется присоединенной функцией Лежандра $m$ го порядка.

Согласно общей теореме присоединенные функции $P^{(m)}_n$ образуют ортогональную систему. Тогда квадрат нормы присоединенних функций равен:

\[
||P^{(m)}_n||^2 = \frac{2}{2n+1}\frac{(n+m)!}{(n-m)!}
\]

