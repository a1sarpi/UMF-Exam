\subsection{Классификация линейных уравнений с частными производными 2-го порядка. Характеристическое уравнение. Приведение уравнения с частными производными к каноническому виду.}
\label{sec:types}

%TODO: Характеристическое уравнение
%autor: Сеня

\paragraph{Классификация уравнений второго порядка.}
Уравнением с частными производными 2-го порядка называется:\footnote{Т.-С. стр. 15 и далее}
\begin{equation}
	F(x, y, u, u_x, u_y, u_{xx}, u_{xy}, u_{yy}) = 0.
\end{equation}

Уравнение называется линейным относительно старших производных, если оно имеет вид
\begin{equation} \label{SOlineq}
	a_{11}(x, y) u_{xx} + 2 a_{12}(x, y) u_{xy} + a_{22}(x, y) y_{yy} + F_1(x, y, u, u_x, u_y) = 0.
\end{equation}
\textit{Линейное} уравнение линейно относительно всех вхождений $u$. 

Линейные уравнения с частными производными 2-го порядка называются \textit{квазилинейными}, если коэффициенты $a_{11}, a_{12}, a_{22}$ зависят от $x, y, u, u_x, u_y$. 

\paragraph{Приведение уравнения к каноническому виду.}
%TODO: Сократить
С помощью преобразования переменных
\begin{equation*}
	\xi = \varphi(x, y), \quad \eta = \psi(x, y),
\end{equation*}
допускающего обратное преобразование, мы получаем новое уравнение, эквиввалентное исходному. Как выбрать $\xi$ и $\eta$ так, тчобы получить наиболее простой вид?

Получим ответ на поставленный вопрос для \eqref{SOlineq}. Преобразуя производные к новым переменным, получаем
\begin{equation} \label{varchange}
	\begin{rcases}
		u_x = u_{\xi} \xi_{x} + u_{\eta} \eta_{x}, \\
		u_y = u_{\xi} \xi_{y} + u_{\eta} \eta_{y}, \\
		u_{xx} = u_{\xi\xi} \xi_{x}^2 + 2 u_{\xi\eta} \xi_{x} \eta_{x} + u_{\eta\eta} \eta^2_{x} + u_{\xi} \xi_{xx} + u_{\eta} \eta_{xx}, \\
		u_{xy} = u_{\xi\xi} \xi_{x} \xi_{y} + u_{\xi\eta}(\xi_x \eta_y + \xi_y \eta_x) + u_{\eta\eta} \eta_x \eta_y + u_{\xi} \xi_{xy} + u_{\eta} \eta_{xy}, \\
		u_{yy} = u_{\xi\xi} \xi_{y}^2 + 2 u_{\xi\eta} \xi_{y} \eta_{y} + u_{\eta\eta} \eta^2_{y} + u_{\xi} \xi_{yy} + u_{\eta} \eta_{yy},
	\end{rcases}
\end{equation}
Подставляя значения производных из \eqref{varchange} в уравнение \eqref{SOlineq}, будем иметь 
\begin{equation} \label{SOeqCHANGED}
	\bar{a}_{11} u_{\xi\xi} + 2 \bar{a}_{12} u_{\xi\eta} + \bar{a}_{22} u_{\eta\eta} + \bar{F} = 0,
\end{equation}
где 
\begin{align*}
	&\bar{a}_{11} = a_{11} \xi^2_{x} + 2 a_{12} \xi_x \xi_y + a_{22} \xi^2_y, \\
	&\bar{a}_{12} = a_{11} \xi_x \eta_x + a_{12} (\xi_x \eta_y + \eta_x \xi_y) + a_{22} \xi_{y} \eta_{y}, \\
	&\bar{a}_{22} = a_{11} \eta^2_{x} + 2 a_{12} \eta_x \eta_y + a_{22} \eta^2_y,
\end{align*}
а функция $\bar{F}$ не зависит от вторых производных. Заметим, что если исходное уравнение линейно, т.е. 
\begin{equation*}
	F(x, y, u, u_x, u_y) = b_1 u_x + b_2 u_y + c u + f,
\end{equation*}
то $\bar{F}$ имеет вид 
\begin{equation*}
	\bar{F}(\xi, \eta, u, u_{\xi}, u_{\eta}) = \beta_1 u_{\xi} + \beta_2 u_{\eta} + \gamma u + \delta,
\end{equation*}
т.е. уравнение остается линейным. 

Выберем переменные $\xi$ и $\eta$ так, чтобы коэффициент $\bar{a}_{11}$ был равен нулю. Рассмотрим уравнение с частными производными 1-го порядка
\begin{equation} \label{FOeq}
	a_11 z^2_x + 2 a_{12} z_x z_y + a_{22} z_y^2 = 0.
\end{equation}
Пусть $z = \varphi(x, y)$ --- какое-нибудь частное решение этого уравнения. Если положить $\xi = \varphi(x, y)$, то коэффициент $\bar{a}_{11}$, очевидно, будет равен нулю Таким образом, упомянутая выше задача о выборе новых независимых переменных связана с решением уравнения \eqref{FOeq}.  

Если $z = \varphi(x, y)$ является частным решением уравнения \eqref{FOeq}, то соотношение $\varphi(x, y) = C$ представляет собой общий интеграл обыкновенного дифференциального уравнения 
\begin{equation} \label{chareq}
	a_{11} dy^2 - 2 a_{12} dx dy + a_{22} dx^2 = 0.
\end{equation}

Если $\varphi(x, y) = C$ представляет собой общий интеграл обыкновенного дифференциального уравнения \eqref{chareq}, то функция $z = \varphi(x, y)$ удовлетворяет уравнению \eqref{FOeq}. 

Уравнение \eqref{chareq} называется \textit{характеристическим} для уравнения \eqref{SOlineq}, а его интегралы - \textit{характеристиками}. 

Пологая $\xi = \varphi(x, y)$, где $\varphi(x, y) = const$ есть общий интеграл уравнения \eqref{chareq}, мы обращаем в нуль коэффициент при $u_{\xi\xi}$. Если $\psi(x, y) = const$ является другим общим интегралом уравнения \eqref{chareq}, независимым от $\varphi(x, y)$, то, пологая $\eta = \psi(x, y)$, мы обратим в нуль также и коэффициент при $u_{\eta\eta}$. 

Уравнение \eqref{chareq} распадается на два уравнения:
\begin{align}
	\frac{dy}{dx} = \frac{a_{12} + \sqrt{a_{12}^2 - a_{11} a_{22}}}{a_{11}}, \\
	\frac{dy}{dx} = \frac{a_{12} - \sqrt{a_{12}^2 - a_{11} a_{22}}}{a_{11}}.
\end{align}

Знак подкоренного выражения определяет тип уравнения \eqref{SOlineq}
\begin{equation*}
	a_{11} u_{xx} + 2 a_{12} u_{xy} + a_{22} u_{yy} + F = 0.
\end{equation*}

Это уравнение мы будем называть в точке $M$ уравнением:
\begin{align*}
	&\texttt{гиперболического } \text{типа, если в точке } M \, \, a_{12}^2 - a_{11} a_{22} > 0, \\
	&\texttt{параболического } \text{типа, если в точке } M \,\, a_{12}^2 - a_{11} a_{22} = 0, \\
	&\texttt{эллиптического } \text{типа, если в точке } M \,\, a_{12}^2 - a_{11} a_{22} < 0.
\end{align*}
Нетрудно убедиться в правильности соотношения
\begin{equation*}
	\bar{a}_{12}^{2} - \bar{a}_{11} \bar{a}_{22} = (a_{12}^{2} - a_{11} a_{22}) D^2, \quad D = \xi_{x} \eta_{y} - \eta_{x} \xi_{y},
\end{equation*}
из которого следует инвариантность типа уравнения при преобразовании переменных, так как функциональный определитель (якобиан) $D$ преобразования переменных отличен от нуля. Вразличных точках области определения уравнение может принадлежать различным типам. 

Рассмотрим область $G$, во всех точках которой уравнение имеет один и тот же тип. Через каждую точку области $G$ проходят две характеристики, причем для уравнений гиперболического типа характеристики действительны и различны, для уравнений эллиптического типа - комплексны и различны, а для уравнений параболического типа обе характеристики действительны и совпадают между собой. 

Для каждого из типов можно вывести каноническую форму уравнения.
\begin{enumerate}
	\item Каноническая форма уравнений гиперболического типа ($a_{12}^2 - a_{11} a_{22} > 0$) \begin{equation}
		u_{xx} - u_{yy} = \Phi \text{ или } u_{xy} = \Phi.
	\end{equation}
	
	\item Для уравнений параболического типа ($a_{12}^2 - a_{11} a_{22} = 0$)
	\begin{equation}
		u_{xx} = \Phi.
	\end{equation}
	
	\item Для уравнений эллиптического типа ($a_{12}^2 - a_{11} a_{22} < 0$)
	\begin{equation}
		u_{xx} + u_{yy} = \Phi.
	\end{equation} 
	
	Во всех случаях $\Phi = -\frac{\bar{F}}{\bar{a}_{22}}$.
\end{enumerate}
