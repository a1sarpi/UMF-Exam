\subsection{Метод разделения переменных решения первой краевой задачи для уравнения Лапласа внутри круга и вне круга. Интеграл Пуассона}
\subsubsection{Лаплас}
%ref: С-Т, стр. 328 (основное); Пикулин, стр. 18
Рассмотрим уравнение Лапласа $ \Delta u = 0 $ внутри круга радиуса $ \rho $ с граничным условием
$ u(\rho, \varphi) = f(\varphi) $. Некоторые дополнительные условия: $ u $ непрерывна внутри круга, $ f
$ непрерывно дифференцируема. От условия на $ f $ можно избавиться.

Перейдём в полярные координаты с центром в центре круга. Там\footnote{См. раздел
\ref{sec:14}.}
\[
    \Delta u = \frac{1}{r} \frac{\partial}{\partial r}\left( r \frac{\partial
    u}{\partial r} \right) + \frac{1}{r^2}\frac{\partial^2 u}{\partial
  \varphi^2} = 0.
\]
Разделим переменные и после подстановки получим
\[
    \frac{(rR')'}{R/r} = - \frac{\Phi''}{\Phi} = \lambda \geqslant 0.
\]
Соответствующая задача Штурма -- Лиувилля $ \Phi'' + \lambda\Phi = 0 $
тогда будет иметь тригонометрические решения\footnote{См. раздел \ref{sec:prost_sl}, <<простейший
случай>>.} $ \Phi = A\cos\sqrt\lambda \varphi + B\sin\sqrt\lambda \varphi $.
При этом условия начальной системы функции $ \Phi $ не касаются, однако из
однозначности решения вытекает $ 2\pi $-периодичность функции $ \Phi(\varphi)
$\footnote{Отсюда и требование к знаку $ \lambda $. Заметим, что случай
$ \lambda = 0 $, $\Phi \equiv \mathrm{const} = A $ входит в обозрение.}.
Это условие можно записать официально как\footnote{Говорить о периодичности
  здесь не совсем корректно, поскольку и так $ \varphi \in [0, 2\pi) $.
Требуется лишь, чтобы для любого $ n $ $ \Phi^{(n)}(0) = \Phi^{(n)}(2\pi) $.
Тогда после возвращения в декартовы координаты функция $ u $ будет непрерывной
со всеми своими производными.} $ \Phi(0) = \Phi(2\pi) $, $ \Phi'(0) =
\Phi'(2\pi) $. Из самого уравнения вытекает тогда, что $ \Phi^{(n)}(0) =
\Phi^{(n)}(2\pi) $ для любого $ n $. После подстановки $ \Phi $ в условия спектр
быстро угадывается, однако можно и строго прийти к уравнению\footnote{См. раздел
\ref{sec:S-L}.} 
\[
  \left(\cos ( \sqrt\lambda 2\pi ) - 1\right)^2 + \sin^2 \left(\sqrt\lambda2\pi\right)=0,
\]
откуда уже явно видно, что $ \lambda_n = n^2 $.

Функцию $ R(r) $, как решение уравнения Эйлера, будем искать в виде $ R(r) = r^\mu $.
Подставляя и сокращая, получаем $ n^2 = \mu^2 $, а $ R(r) = Cr^n + Dr^{-n} +
E\ln r $.
Теперь, исходя из ограниченности гармонических функций и условий задачи, приравниваем один из
коэффициентов $ C $, $ D $, а также коэффициент $ E $ к нулю и получаем частные решения  
\begin{align*}
  u_n(r,\varphi) &= r^n(A_n\cos n\varphi + B_n \sin n\varphi) \quad \text{для
  задачи внутри круга},\\
    u_n(r,\varphi) &= r^{-n}(A_n\cos n\varphi + B_n \sin n\varphi) \quad \text{для
  задачи вне круга}.\\
\end{align*}
Учитывая получившийся вид решения, разложим граничное условие $ f(\varphi) $ в
ряд Фурье 
\begin{equation}
    \alpha_0 = \frac{1}{\pi}\int\limits_{-\pi}^{\pi}f(\psi)\,d\psi, \quad
    \alpha_n = \frac{1}{\pi}\int\limits_{-\pi}^{\pi}f(\psi)\cos
    n\psi\,d\psi,\quad
    \beta_n = \frac{1}{\pi}\int\limits_{-\pi}^{\pi}f(\psi)\sin n\psi\,d\psi
    \label{eq:fourier_exp}
\end{equation}
и приравняем коэффициенты ряда $ u(a, \varphi) $, откуда уже получим $ A_n $ и $ B_n $.
Конкретно, 
\begin{align}
  f(\varphi) &= \frac{\alpha_0}{2} + \sum_{n=1}^\infty(\alpha_n\cos n\varphi +
  \beta_n \sin n\varphi),\\
  u(\rho, \varphi) &= \frac{\alpha_0}{2} + \sum_{n=1}^\infty t^n (\alpha_n\cos
  n\varphi + \beta_n \sin n\varphi), \label{eq:fourier_sol}
\end{align}
где $ t = r/a \leqslant 1 $ для внутренней задачи и
$ t = a/r \leqslant 1 $ для внешней.

%TODO: как раскладывать в ряд Фурье (?)
%TODO: формулы для An Bn (?)



\subsubsection{Интеграл Пуассона}
%ref: С-Т, стр. 333; Пикулин, стр. 20 (основное); Олейник, стр. 86; Шубин, стр. 130;
%Свешников, стр. 212; Петровский, стр. 251
Ограничимся на время внутренним случаем и соединим формулы
\eqref{eq:fourier_exp} и \eqref{eq:fourier_sol}:
\[
  u(r,\varphi) = \frac{1}{\pi} \int\limits_{-\pi}^{\pi} \left[ \frac{1}{2} + \sum_{n=1}^\infty \left(
  \frac{r}{a} \right)^n (\cos n\psi\cos n\varphi + \sin n\psi \sin n\varphi
)\right] \, d\psi.
\]
Используя формулы косинуса суммы, Эйлера, суммы геометрической прогрессии,
комплексную формулу косинуса, при
$ t := r/a < 1 $ получим
\begin{multline*}
    \frac{1}{2} + \sum t^n(\cos n\psi \cos n\varphi + \sin n\psi \sin n\varphi)
    = \frac{1}{2} + \sum t^n \cos n(\varphi - \psi) =\\=
    \frac{1}{2} + \frac{1}{2}\sum t^n \left[ e^{in(\varphi-\psi)} +
    e^{-in(\varphi - \psi)} \right] = \frac{1}{2} \left( 1 + \sum \left[
(te^{i(\varphi - \psi)})^n + (te^{-i(\varphi - \psi)})^n \right]  \right) =\\=
  \frac{1}{2} \left[ 1 + \frac{te^{i(\varphi - \psi)}}{1 - te^{i(\varphi-\psi)}}
    + \frac{te^{-i(\varphi - \psi)}}{1 - te^{-i(\varphi-\psi)}}\right] =
    \frac{1}{2} \frac{1 - t^2}{1 - 2t\cos(\varphi - \psi) + t^2}.
\end{multline*}
Таким образом, \emph{интегралом Пуассона} называется интеграл 
\[
    \frac{1}{2\pi} \int\limits_{-\pi}^{\pi}f(\psi) K(r, \varphi, a,
    \psi)\,d\psi,
\]
а выражение  
\[
    K(r,\varphi, a,\psi) = \frac{a^2 - r^2}{r^2 - 2ar\cos(\varpsi-\psi) + a^2}
\]
называют \emph{ядром Пуассона}. Видно, что $ K > 0 $, поскольку при $ r < a $ и
$ 2ar < a^2 + r^2 $. При $ r = a $ интеграл теряет смысл, однако в пределе при
$ r \to a $ стремится\footnote{Поскольку ряд, из
которого был получен интеграл, непрерывен в замкнутой области.} к граничной функции $ f(\varphi) $, что позволяет
доопределить его соответствующим образом и назвать решением $
u(r,\varphi) $.

Понятно, что для внешней задачи надо просто
поменять числитель ядра на $ r^2 - a^2 $.
