\subsection{Пространство быстроубывающих функций и пространство функций медленного роста.
Обобщённое преобразование Фурье. Обобщённое преобразование Фурье свертки и обобщённое равенство
Парсеваля.}

\subsubsection{Пространство функций медленного роста}

ОПРЕДЕЛЕНИЕ. Пространство основных функций $\mathcal{S}$ определим следующим образом: это все
функции класса $\mathcal{C}^\infty (\mathbb{R}^n)$, убывающие при $|x| \to \infty$ вместе со всеми
производными быстрее $|x|^{-k}$. Сходимость в $\mathcal{S}$ определим следующим образом:
последовательность функций $\varphi_k \in \mathcal{S}$ сходится к функции
$\varphi \in \mathcal{S}, \varphi_k \to \varphi, k \to \infty \text{ в } \mathcal{S}$, если для
всех $\alpha, \beta$
\[
  x^\beta \partial^\alpha \varphi_k (x) \rightrightarrows x^\beta \partial^\alpha \varphi(x), x\in\mathbb{R}^n, k \to \infty.
\]

Это пространство получается замкнутым относительно дифференцирования:
$\forall \varphi \in \mathcal{S} \forall \alpha : \partial^\alpha \varphi \in \mathcal{S}$.
Но что было хорошо с пространством $\mathcal{D}$, так это то, что оно было замкнуто относительно
умножения на любую бесконечно-дифференцируемую функцию:
$\forall \varphi \in \mathcal{D} \forall f \in \mathcal{C}^\infty (\mathbb{R}^n) : \varphi(x) \cdot f(x) \in \mathcal{D}$. Такого нельзя сказать про пространство $\mathcal{S}$, однако если чуть
сузить класс функций, на которые мы хотим уметь умножать, то получим пространство функций медленного
роста (????):
\[
  \Omega_M = \left\{ a\in\mathcal{C}^\infty(\mathbb{R}^n) :
    \forall\alpha : \left| \partial^\alpha a(x) \right|
      \leqslant C_\alpha (1+|x|)^{m(\alpha)} \right\} 
\]

ОПРЕДЕЛЕНИЕ. Пространством обобщённых функций медленного роста называется пространство линейных
непрерывных функционалов над $\mathcal{S}$.

Верна следующая теорема:
\begin{theorem}[Л. Шварц]
  Для того, чтобы линейный функционал $f$ на $\mathcal{S}$ принадлежал $\mathcal{S}'$ (т.е. был
  непрерывным на $\mathcal{S}$), необходимо и достаточно, чтобы существовали число $C>0$ и целое
  число $p\geqslant 0$ такие, что
  \[
    |(f, \varphi)| \leqslant C \|\varphi\|_p
  \]
  для любой $\varphi \in \mathcal{S}$, где
  \[
    \|\varphi\|_p = \sup_{|\alpha| \leqslant p, x\in\mathbb{R}^n} (1+|x|)^p |\partial^\alpha \varphi(x)|.
  \]
\end{theorem}

Пользуясь теоремой Л. Шварца, можно доказать, что всякая обобщенная функция из $\mathcal{S}'$
является производной (в смысле обобщённых функций) от непрерывной функции медленного роста. Этим и
объясняется название пространства $\mathcal{S}'$.

\subsubsection{Преобразование Фурье обобщённых функций из $\mathcal{S}'$}

Определим преобразование Фурье обобщённой функции так, чтобы выполнялось соотношение
\[
  (F[f], \varphi) = (f, F[\varphi]), \varphi \in \mathcal{S}.
\]
делаем именно так, чтобы если подставить вместо $f$ что-то обычное, получалось бы верное равенство:
\begin{multline*}
  \int F[f] \varphi \, d\xi
  = \int \left( \int f(x) e^{i (\xi, x)} \, dx \right) \varphi(\xi) \, d\xi
  = \int f(x) \left( \int \varphi(\xi) e^{i (x, \xi)} \, d\xi \right) \, dx
  = \int f(x) F[\varphi](x) \, dx.
\end{multline*}

Введём на $\mathcal{S}'$ еще операцию $F^{-1}$:
\[
  F^{-1}[f] = \dfrac{1}{(2\pi)^n} F[f(-x)], f\in\mathcal{S}'.
\]

Из важного:
\begin{itemize}
  \item 
    \[
      \partial^\alpha F[f] = F \left[ (ix)^\alpha f(x) \right].
    \]
  \item
    \[
      F \left[ \partial^\alpha f \right] (\xi) = (-i\xi)^\alpha F[f] (\xi).
    \]
  \item Преобразование Фурье свёртки
    \[
      F \left[ f * g \right] = F[f] \cdot F[g]
    \]
\end{itemize}

\subsubsection{Равенство Парсеваля}

% TODO дописать равенство Парсеваля

