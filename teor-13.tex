\subsection{Функции, гармонические в области. Теорема о среднем значении для гармонических
функций. Принцип максимума.}

\paragraph{Гармонические в области функции}
ОПРЕДЕЛЕНИЕ. Вещественнозначная функция $u(x)$ класса $\mathcal{C}^2 (G)$ называется 
\emph{гармонической в области $G$}, если она удовлетворяет уравнению Лапласа $\Delta u = 0$
в этой области.

При $n=1$ гармонические фунцкции своядтся к линейным функциям, и потом их теория интереса
не представляет. Поэтому в дальнейшем будем считать $n \geqslant 2$. Нетривиальным примером
гармонической функции при $\| \vec{x} \| \neq 0$ является фундаментальное решение оператора
Лапласа:
\begin{align*}
  \varepsilon_2 (x) &= \dfrac{1}{2\pi} \ln \| \vec{x} \|, n = 2, \\
  \varepsilon_n (x) &= - \dfrac{1}{(n-2) \sigma_n} \| \vec{x} \|^{-n+2}, n \geqslant 3,
\end{align*}
этот факт легко доказать, если перейти к обобщённым сферическим координатам (ну а для малых
размерностей это будет сооствественно полярные и обычные сферические). В этих координатах
$\| \vec{x} \| = r$, а все члены оператора Лапласа, зависящие от производных по углам, будут
заведомо равны нулю. Тогда оставшийся член будет равен:
\[
  \Delta \varepsilon_n = 
  \dfrac{\partial^2 \varepsilon_n}{\partial r^2} 
  + \dfrac{n-1}{r} \dfrac{\partial \varepsilon_n}{\partial r}, 
\]
в области $\|\vec{x}\| \neq 0$, очевидно, это равно нулю.

Подставим в первую формулу Грина какую-либо гармоническую функцию $v : \Delta v = 0$, и функцию 
$u = 1$:
\[
  \iiint_T u \Delta v \, d\tau = - \iiint_T \nabla u \cdot \nabla v \, d\tau
    + \iint_\Sigma u \dfrac{\partial v}{\partial n} \, d\sigma
  \Rightarrow
  \iint_\Sigma \dfrac{\partial v}{\partial n} \, d\sigma = 0.
\]

Из этого как следствие возникает необходимое условие существования решения второй краевой задачи
для уравнения Лапласа: задача
\[
  \begin{cases}
    \Delta u(\vec{x}) = 0, \vec{x} \in T, \\
    \dfrac{\partial u}{\partial n} (\vec{x}) = f, \vec{x} \in \Sigma,
  \end{cases}
\]
может иметь решения только при условии $\iint_\Sigma f \, d\sigma = 0$.

\paragraph{Теорема о среднем значении для гармонических функций}
\begin{theorem}
  Если функция $u(M)$ гармонична в некоторой области $T$, а $M_0$ -- какая-нибудь точка, лежащая
  внутри области $T$, то имеет место формула:
  \[
    u(M_0) = \dfrac{1}{4\pi a^2} \iint \limits_{\Sigma_a} u \, d\sigma,
  \]
  где $\Sigma_a$ -- сфера радиуса $a$ с центром в точке $M_0$, целиком лежащая в области $T$
\end{theorem}

Эта теорема утверждает, что значение гармонической функции в нектоторой точке $M_0$ равно
среднему значению этой функции на любой сфере $\Sigma_a$ с цетром в $M_0$, если сфера
$\Sigma_a$ не выходит из области гармоничности функции $u(M)$.

\paragraph{Принцип максимума}

\begin{theorem}
  Если функция $u(M)$, определенная и непрерывная в замкнутой области $T+\Sigma$, удовлетворяет
  уравнению $\Delta u = 0$ внутри $T$, то максимальное и минимальное значения функции $u(M)$
  достигаются на поверхности $\Sigma$.
\end{theorem}

Набросок доказательства: предположим, что максимум достигается в некоторой точке $M_0 \in T$,
окружим эту точку сферой радиуса $\rho$, целиком лежащей внутри области $T$. Тогда
$u(M_0) \geqslant \left. u \right|_{T+\Sigma}$. По теореме о среднем: 
\[
  u(M_0) = \dfrac{1}{4\pi \rho^2} \iint_{\Sigma_\rho} u(M) \, d\sigma \leqslant 
  \dfrac{1}{4\pi \rho^2} \iint_{\Sigma_\rho} u(M_0) \, d\sigma = u(M_0).
\]
Если это максимум, то либо $u(M_0) > u(M)$, тогда в неравенстве выше будет строгий знак, что
означает противоречие, либо $u(M_0) = u(M)$, то есть если максимум и есть внутри области $T$, 
то только в случае константного значения внутри этой области.
