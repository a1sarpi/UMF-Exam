\subsection{Фундаментальное решение дифференциального оператора. Обобщённое решение задачи Коши.}

Написанное ниже взято из Владимирова, страницы 144--147.

Обозначения:
\begin{itemize}
  \item $\mathcal{D}$ -- пространство (обычных) основных функций: финитные бесконечно
    дифференцируемые в $\mathbb{R}^n$ функции; 
  \item $\mathcal{D}'$ -- пространство (обычных) обобщённых функций;
  \item $\mathcal{S}$ -- пространство основных функций медленного роста: все бесконечно
    дифференцируемые функции на $\mathbb{R}^n$, убывающие при $|x| \to \infty$ вместе со
    всеми производными быстрее любой степени $|x|^{-1}$;
  \item $\mathcal{S}'$ -- пространство обобщённых функций медленного роста;
\end{itemize}

\paragraph{Фундаментальное решение}
ОПРЕДЕЛЕНИЕ. Пусть $L$ -- дифференциальный оператор с постоянными коэффициентами,
$a_\alpha (x) = a_\alpha = \operatorname{const}$,
\[
  L(\partial) = \sum_{|\alpha| = 0}^m a_\alpha \partial^\alpha, L^* (\partial) = L(-\partial).
\]
\emph{Фундаментальным решением (функцией влияния)} оператора $L(\partial)$ называется обобщенная
функция $\mathcal{E} \in \mathcal{D}' (\mathbb{R}^n)$, удовлетворяющая в $\mathbb{R}^n$ 
уравнению
\[
  L(\partial) \mathcal{E} = \delta (x).
\]

Оно не единственно и определяется с точностью до слагаемого
$\mathcal{E}_0: L(\partial) \mathcal{E}_0 = 0$.

Верна следующая лемма:
\begin{theorem}
  \[
    \mathcal{E} \in \mathcal{S}' \text{-- фундаментальное решение $L(\partial)$}
    \Leftrightarrow
    L(-i \xi) F[\mathcal{E}] = 1,
  \]
  где $L(\xi) = \sum_{|\alpha|=0}^m a_\alpha \xi^\alpha$, $F[\mathcal{E}]$ -- преобразование Фурье.
\end{theorem}

Эта лемма следует из свойств преобразования Фурье. Из этой леммы мы знаем о том, каким образом можно
находить фундаментальные решения, а так же знаем, что они не единственны в силу не единственности 
решений получаемых алгебраических уравнений. Так же было доказано, что получаемое уравнение всегда разрешимо в классе $\mathcal{S}'$, если $L(-i\xi)$ не равно тождественно нулю.


\begin{theorem}[Теорема о решении уравнения с правой частью]
  Пусть $f \in \mathcal{D}'$ такова, что $\exists \mathcal{E} * f \in \mathcal{D}'$. Тогда решение
  уравнения $L(\partial) u = f(x)$ существует в $\mathcal{D}'$ и даётся формулой
  \[
    u = \mathcal{E} * f.
  \]
  Причём это решение единственно в классе тех обобщённых функций из $\mathcal{D}'$, для которых
  существует свёртка с $\mathcal{E}$.
\end{theorem}
Эта теорема следует из свойств дифференцирования свёртки.



\paragraph{Обобщённое решение задачи Коши} \footnote{(Владимиров, стр 168)}
Сформулируем задачу Коши: оператор $L = \sum_{k=0}^m a_k \dfrac{d}{dx^k}$:
\[
  \begin{cases}
    Ly = f(x), x > 0, f \in \mathcal{C} \\
    y(0) = a_0, \\
    y'(0) = a_1, \\
    \dots \\
    y^{(m-1)} (0) = a_{(m-1)}.
  \end{cases}
\]
% TODO недописано



Обозначим за $Z(x)$ решение однородного уравнения:
\[
  \begin{cases}
    LZ = 0, \\
    Z(0) = Z'(0) = \dots = Z^{(m-2)} (0) = 0, \\
    Z^{(m-1)} (0) = 1.
  \end{cases}
\]
(на всякий случай, такое решение будет в виде: $Z(x) = \sum e^{\lambda_k x}$)

Покажем, что 


Так же вот табличка основных фундаментальных решений:
\begin{center}
  \begin{tabular}{|c|c|}
    \hline
    Оператор & фундаментальное решение \\
    \hline
    теплопроводности $\dfrac{\partial \mathcal{E}}{\partial t} - a^2 \Delta \mathcal{E}$ &
    $\mathcal{E} (x, t) = \dfrac{\theta(t)}{(2a \sqrt{\pi t})^n} \exp \left\{ -\dfrac{|x|^2}{4a^2 t} \right\}$ \\
    
    \hline
    Лаплас $\Delta \mathcal{E}$ &
    $\mathcal{E}_2 = \dfrac{1}{2\pi} \ln |x|, \mathcal{E} = -\dfrac{1}{(n-2) \sigma_n} |x|^{-n+2}$ \\
    \hline 
    Гельмгольца $(\Delta + k^2) \mathcal{E}_n$ &
    $\mathcal{E}_1 (x) = \dfrac{1}{2ik} e^{ik |x|}, \bar\mathcal{E}_1 (x) = - \dfrac{1}{2ik} e^{-ik|x|}$ \\
    \hline
  \end{tabular}
\end{center}


