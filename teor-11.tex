\subsection{Уравнение теплопроводности/диффузии на бесконечной и полубесконечной прямой.
Функция влияния мгновенного точечного источника.}

\paragraph{Бесконечная прямая}
\begin{equation}\label{1.10.1-main}
  \begin{cases}
    u_t = a^2 u_{xx} + f(x, t), &-\infty < x < +\infty, t > 0, \\
    |u(\pm \infty, t)| < \infty, \\
    u(x, 0) = \varphi(x),
  \end{cases}
\end{equation}
а теперь проделаем разделение переменных, причём <<переменную разделения>> обозначим за $\lambda^2$,
потому что несложно проверить, что при отрицательных значениях не будет органиченности на
бесконечности. После разделения переменных получаем такую задачу для $X(x)$:
\[
  \begin{cases}
    -X'' - \lambda^2 X = 0, \\
    |X(\pm \infty)| < \infty,
  \end{cases}
\]
несложно получить, что собственные значения такой задачи имеют непрерывный спектр
$\lambda \in (-\infty, +\infty)$, а собственные функции: $X = e^{i\lambda x}$.

Это всё позволяет сказать о общем виде будущего решения:
\[
  u(x, t) = \int\limits_{-\infty}^{+\infty} A(t, \lambda) e^{i \lambda x} \, d\lambda,
\]
подставим это в \eqref{1.10.1-main}:
\[
  \begin{cases}
    \int\limits_{-\infty}^{+\infty} \dfrac{\partial A(t, \lambda)}{\partial t} e^{i \lambda x} \, d\lambda
      = \int\limits_{-\infty}^{+\infty} - a^2 \lambda^2 A(t, \lambda) e^{i \lambda x} \, d\lambda
      + \int\limits_{-\infty}^{+\infty} F(t, \lambda) e^{i \lambda x} \, d\lambda, \\
    
    \int\limits_{-\infty}^{+\infty} A(0, \lambda) e^{i \lambda x} \, d\lambda 
      = \int\limits_{-\infty}^{+\infty} \Phi(\lambda) e^{i \lambda x} \, d\lambda,
  \end{cases}
\]
где $F, \Phi$ -- соответсвующие коэффициенты разложения по базису, а по совместительству -- образы 
преобразования Фурье.

\[
  \begin{cases}
    \dfrac{\partial A(t, \lambda)}{\partial t}
      = - a^2 \lambda^2 A(t, \lambda)
      + F(t, \lambda), \\
    A(0, \lambda) = \Phi(\lambda),
  \end{cases}
\]
такой диффур решим, аналогично такому же диффуру в \ref{question:split-variables-in-heat}, то
есть методом вариации произвольных постоянных. Тогда ответ получим в виде:
\[
  A(t, \lambda) = \left(
    \Phi(\lambda)
    + \int\limits_0^t F(\tau, \lambda) e^{a^2 \lambda^2 \tau} \, d\tau \right)
    \cdot e^{- a^2 \lambda^2 t}
\]
теперь всё что остаётся это найти $F$ и $\Phi$, а их можно найти с помощью обратного преобразования
Фурье:
\[
  f(x, t) = \int\limits_{-\infty}^{+\infty} F(t, \lambda) e^{i \lambda x} \, d\lambda
  \Leftrightarrow
  F(t, \lambda)
  = \dfrac{1}{2\pi} \int\limits_{-\infty}^{+\infty} f(\xi, t) e^{-i \lambda \xi} \, d\xi,
\]
и аналогично для $\Phi$. Тогда окончательный ответ перепишем в виде:
\begin{multline*}
  u(x, t) = \int\limits_{-\infty}^{+\infty}
    \left( \dfrac{1}{2\pi} \int\limits_{-\infty}^{+\infty} \varphi(\xi) e^{-i \lambda \xi} \, d\xi
      + \dfrac{1}{2\pi} \int\limits_0^t
      \left( \int\limits_{-\infty}^{+\infty} f(\xi, \tau) e^{-i \lambda \xi} \, d\xi \right)
      e^{a^2 \lambda^2 \tau} d\tau \right)
    \cdot e^{-a^2 \lambda^2 t} \cdot e^{i \lambda x} \, d\lambda = \\
  = \int\limits_{-\infty}^{+\infty}
    \left( \int\limits_{-\infty}^{+\infty}
      \dfrac{1}{2\pi} e^{-i\lambda (\xi-x) - a^2\lambda^2t} \, d\lambda \right) \varphi(\xi) \, d\xi
  + \int\limits_{-\infty}^{+\infty} \int\limits_{0}^{t}
    \left( \dfrac{1}{2\pi} e^{-i\lambda(\xi-x) - a^2\lambda^2(t - \tau)} \right)
    f(\xi, \tau) \, d\tau d\xi = \\
  = \int\limits_{-\infty}^{+\infty} G(x, \xi, t) \varphi(\xi) \, d\xi
    + \int\limits_0^t \int\limits_{-\infty}^{+\infty} G(x, \xi, \tau) f(\xi, t-\tau) \, d\xi d\tau.
\end{multline*}

Функция влияния мгновенного точечного источника:
\[
  G(x, \xi, t) = \dfrac{1}{2\pi} \int\limits_{-\infty}^{+\infty} e^{-i \lambda(\xi-x) - a^2 \lambda^2 t} \, d\lambda
  = \dfrac{1}{2\sqrt{\pi a^2 t}} e^{- \dfrac{(x-\xi)^2}{4 a^2 t}}
\]
Причём эта функция сама удовлетворяет уравнению теплопроводности по переменным $x$ и $t$.
% Следующий этап будет разложение функций $f(x, t), \varphi(x)$ в ряды Фурье. Но у нас получился
% непрерывный спектр, то есть это будет не ряд, а интеграл:
% \[
%   f(x, t) = \int\limits_{-\infty}^{+\infty} F_{\lambda} (t) e^{i \lambda x} \, d\lambda,
%   \varphi(x) = \int\limits_{-\infty}^{+\infty} \Phi_\lambda e^{i \lambda x} \, d\lambda,
% \]
% возьмём обратное преобразование Фурье от обоих частей и получим:
% \[
  
% \]


\paragraph{Полубесконечная прямая}
Аналогично предыдущему вопросу рассматриваем задачу:
\begin{equation}\label{1.10.2-main}
  \begin{cases}
    u_t = a^2 u_{xx} + f(x, t), &0 < x < +\infty, t > 0, \\
    \alpha u_x(0, t) + \beta u(0, t) = 0, \\
    |u(+\infty, t)| < \infty, \\
    u(x, 0) = \psi(x),
  \end{cases}
\end{equation}

Метод, классически используемый для решения таких задач называется методом продолжения. Идея этого
метода состоит в том, чтобы перейти к уже решённой задаче для бесконечной прямой. Рассмотрим этот
метод для задачи без неоднородности:
\[
  \begin{cases}
    u_t = a^2 u_{xx}, &x > 0, t > 0, \\
    \alpha u_x(0, t) + \beta u(0, t) = 0, \\
    |u(+\infty, t)| < \infty, \\
    u(x, 0) = \psi(x),
  \end{cases}
\]
Итак, перейдём к задаче на функцию $U(x, t)$, определённую на $x \in \mathbb{R}, t > 0$, причём
искомое решение $u(x, t)$ будет просто выражаться как сужение $u(x, t) = U(x, t), x\geqslant 0, t>0$.
Тогда задача на $U$ будет выглядеть как-то так:
\[
  \begin{cases}
    U_t = a^2 U_{xx}, &t > 0, \\
    \alpha U_x(0, t) + \beta U(0, t) = 0, \\
    |U(\pm \infty, t)| < \infty, \\
    U(x, 0) = \Psi(x),
  \end{cases}
\]
и здесь не понятно, каким образом продолжить функцию $\psi$. Тогда решим задачу с конца, раз уж мы
знаем общий вид решения задачи на бесконечной прямой:
\[
  U(x, t) = \int\limits_{-\infty}^{+\infty} G(x, \xi, t) \cdot \Psi(\xi) \, d\xi, \quad
  G(x, \xi, t) = \dfrac{1}{2\sqrt{\pi a^2 t}} e^{- \dfrac{(x-\xi)^2}{4 a^2 t}},
\]
подставим этот вид в граничное условие в нуле (и еще используем тот факт, что $\dfrac{\partial G}{\partial x} = - \dfrac{\partial G}{\partial \xi}$):
{ \small
\begin{multline*}
  \left. 
    \alpha \int\limits_{-\infty}^{+\infty}
      \dfrac{\partial G(x, \xi, t)}{\partial x} \Psi(\xi) \, d\xi
    + \beta \int\limits_{-\infty}^{+\infty} G \Psi(\xi) \, d\xi \right|_{x=0} = \\
  = \left. 
    - \alpha \int\limits_{-\infty}^{+\infty}
      \dfrac{\partial G(x, \xi, t)}{\partial \xi} \Psi(\xi) \, d\xi
    + \beta \int\limits_{-\infty}^{+\infty} G \Psi(\xi) \, d\xi \right|_{x=0} = \\
  = \left. 
    \left. - \alpha G(x, \xi, t) \Psi(\xi) \right|_{-\infty}^{+\infty}
    + \alpha \int\limits_{-\infty}^{+\infty} G(x, \xi, t) \Psi' (\xi) \, d\xi
    +\beta \int\limits_{-\infty}^{+\infty} G(x, \xi, t) \Psi(\xi) \, d\xi \right|_{x=0} = \\
  =  \int\limits_{-\infty}^{+\infty} 
      \left( \alpha \Psi' (\xi) + \beta \Psi(\xi) \right)
      G(0, \xi, t) \, d\xi = \\
  = \int\limits_{0}^{+\infty} 
      \left( \alpha \Psi' (\xi) + \beta \Psi(\xi) \right)
      G(0, \xi, t) \, d\xi
    + \int\limits_{-\infty}^{0} 
      \left( \alpha \Psi' (\zeta) + \beta \Psi(\zeta) \right)
      G(0, \zeta, t) \, d\zeta
  = \left|\, \begin{aligned}
    \xi = -\zeta \\
    d\xi = - d\zeta \\
    G(0, -\xi, t) = G(0, \xi, t)
  \end{aligned} \,\right| = \\
  = \int\limits_{0}^{+\infty} 
      \left( \alpha \Psi' (\xi) + \beta \Psi(\xi) \right)
      G(0, \xi, t) \, d\xi
    + \int\limits_{0}^{+\infty} 
      \left( \alpha \Psi' (-\xi) + \beta \Psi(-\xi) \right)
      G(0, \xi, t) \, d\xi = \\
  = \int\limits_{0}^{+\infty} 
    \left( ( \alpha\Psi'(\xi) + \beta\Psi(\xi)) + (\alpha \Psi'(-\xi) +\beta\Psi(-\xi)) \right)
      G(0, \xi, t) \, d\xi,
\end{multline*}
}
таким образом, если $\Psi$ такая, что $\alpha\Psi' + \beta\Psi$ -- нечётная функция, то выполняется
граничное условие в нуле. При каждом конкретном наборе $\alpha, \beta$ надо опять искать как
продолжать функцию $\psi$. Но для базовых задач Дирихле и Неймана можно использовать готовые решения.

Рассмотрим к примеру задачу Дирихле: $\alpha = 0, \beta = 1$. Тогда: $\Psi(x) = \begin{cases}
  \psi(x), x > 0, \\
  - \psi(-x), x < 0
\end{cases} $. Теперь разберемся с видом решения:
\begin{multline*}
  u(x, t) \stackrel{x>0}{=} U(x, t) = \int\limits_{-\infty}^{+\infty} \Psi(\xi) G(x, \xi, t) \, d\xi
  = \int\limits_{0}^{+\infty} \psi(\xi) G(x, \xi, t) \, d\xi
    - \int\limits_{0}^{+\infty} \psi(\xi) G(x, -\xi, t) \, d\xi
  = \int\limits_{0}^{+\infty} \left( G(x, \xi, t) - G(x, -\xi, t) \right) \psi(\xi) \, d\xi.
\end{multline*}
Таким образом, функция влияния мгновенного точечного источника:
\[
  G_1(x, \xi, t) = G(x, \xi, t) - G(x, -\xi, t)
  = \dfrac{1}{2\sqrt{\pi a^2 t}} \left( e^{- \dfrac{(x-\xi)^2}{4 a^2 t}} - e^{- \dfrac{(x+\xi)^2}{4 a^2 t}} \right) 
\]

И дальше несложно показать, что решение в неоднородной задаче будет выражаться:
\[
  \begin{cases}
    u_t = a^2 u_{xx} + f(x, t), \quad 0 < x < +\infty, \\
    \alpha u_x(0, t) + \beta u(0, t) = 0, \\
    |u(+\infty, t)| < \infty, \\
    u(x, 0) = \psi(x),
  \end{cases}
  \Rightarrow
  u(x, t) = \int\limits_0^{+\infty} \psi(x) G_1(x, \xi, t) \, d\xi
  + \int\limits_{0}^{+\infty} \int_0^t f(\xi, t-\tau) G(x, \xi, t - \tau) \, d\tau d\xi.
\]
