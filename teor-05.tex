\subsection{Энергия колебаний ограниченной струны. Теорема единственности для смешанной краевой задачи для уравнения колебаний струны.}

\paragraph{Энергия колебаний ограниченной струны.}

Найдем выражение для энергии поперечных колебаний струны $E = K + U$, где $K$ --- кинетическая и $U$ - потенциальная энергия. Элемент струны $dx$, движущийся со скоростью $v = u_t$, обладает кинетической энергией 
\begin{equation*}
	\frac{1}{2} m v^2 = \frac{1}{2} \rho(x) (u_t)^2 \, dx \quad (m = \rho \, dx)
\end{equation*}

Кинетическая энергия всей струны равна 
\begin{equation}
	K = \frac{1}{2} \int \limits_{0}^{l} \rho(x) [u_t(x, t)]^2 \, dx.
\end{equation}

Потенциальная энергия поперечных колебаний струны, имеющей при $t = t_0$ форму $u(x, t_0) = u_0(x)$, равна работе, которую надо совершить, чтобы струна перешла из положения равновесия в положение $u_0(x)$. Пусть функция $u(x, t)$ дает профиль струны в момент $t$, причем
\begin{equation*}
	u(x, 0) = 0, \quad u(x, t_0) = u_0(x).
\end{equation*}
Элемент $dx$ под действием равнодействующей сил натяжения 
\begin{equation*}
	T \frac{\partial u}{\partial x} \Big|_{x + dx} - T \frac{\partial u}{\partial x} \Big|_{x} = T u_{xx} \, dx
\end{equation*}
за время $dt$ проходит путь $u_t(x, t) \, dt$. Работа, производимая всей струной за время $dt$, равна
\begin{equation*}
	\Bigg\{\int \limits_{0}^{l} T_0 u_{xx} u_t \, dx \Bigg\} \, dt = \Bigg\{T_0 u_x u_t \Big|_{0}^{l} - \int \limits_{0}^{l} T_0 u_x u_{xt} \, dx \Bigg\} \, dt =  \Bigg\{-\frac{1}{2} \frac{d}{dt} \int \limits_{0}^{l} T_0 (u_x)^2 \, dx + T_0 u_x u_t \Big|_{0}^{l}\Bigg\} \, dt.
\end{equation*}
Интегрируя по $t$ от $0$ до $t_0$, получаем
\begin{equation*}
	-\frac{1}{2} \int \limits_{0}^{l} T_0(u_x)^2 \, dx \Big|_{0}^{t_0} + \int \limits_{0}^{t_0} T_0 u_x u_t \Big|_{0}^{l} \, dt = -\frac{1}{2} \int \limits_{0}^{l} T_0 [u_x(x, t_0)]^2 \, dx + \int \limits_{0}^{t_0} T_0 u_x u_t \Big|_{0}^{l} \, dt.
\end{equation*}

Нетрудно выяснить смысл последнего слагаемого правой части этого равенства. Действительно, $T_0 u_x|_{x = 0}$ есть величина натяжения на конце струны $x = 0; \, u_t(0, t) \, dt$  --- перемещение этого конца, а интеграл 
\begin{equation}
	\int \limits_{0}^{t_0} T_0 u_x u_t|_{x = 0} \, dt
\end{equation}
представляет работу, которую надо затратить на перемещение конца $x = 0$. Аналогичный смысл имеет слагаемое, соответствующее $x = l$. 

Если концы струны закрепленны, то работа на них будет равна нулю (при этом $u(0, t) = 0, u_t(0, t) = 0$). Следовательно, при перемещении закрепленной на концах струны из положения равновесия $u = 0$ в положение $u_0(x)$ работа не зависит от способа перехода струны в это положение и равна 
\begin{equation}
	-\frac{1}{2} \int \limits_{0}^{l} T_0[u'_0(x)]^2 \, dx, 
\end{equation}
потенциальной энергии струны в момент $t = t_0$ с обратным знаком.

Таким образом, полная энергия струны равна
\begin{equation}
	E = \frac{1}{2} \int \limits_{0}^{l} [T_0(u_x)^2 + \rho(x)(u_t)^2] \, dx.
\end{equation}

Совершенно аналогично может быть получено выражение для потенциальной энергии продольных колебаний стержня. Впрочем, его можно получить также, исходя из формулы для потенциальной энергии упругого стержня
\begin{equation*}
	U = \frac{1}{2} k(\frac{l - l_0}{l_0})^2 l_0,
\end{equation*}
где $l_0$ --- начальная длина стержня, $l$ --- конечная длина. Отсюда непосредственно следует 
\begin{equation*}
	U = \frac{1}{2} \int \limits_{0}^{l} k(u_x)^2 \, dx.
\end{equation*}

\paragraph{Теорема единственности для смешанной краевой задачи для уравнения колебаний струны.}

\textit{Возможно существование только одной функции $u(x, t)$, определенной в области $0 \leqslant x \leqslant l, t \geqslant 0$ и удовлетворяющей уравнению}
\begin{align} \label{equa}
	\rho(x) \frac{\partial^2 u}{\partial t^2} = \frac{\partial}{\partial x} \Big(k(x) \frac{\partial u}{\partial x}\Big) + F(x, t) \quad (\rho(x) > 0, k(x) > 0), \\
	0 < x < l, t > 0,
\end{align}
\textit{начальным и граничным условиям}
\begin{equation}
	\begin{rcases}
		u(x, 0) = \varphi(x), \quad u_t(x, 0) = \psi(x), \\
		u(0, t) = \mu_1(t), \quad u(l, t) = \mu_2(t),
	\end{rcases}
\end{equation}
\textit{если выполнены условия:}
\textit{\begin{enumerate}
	\item функция $u(x, t)$ и производные, входящие в уравнение \eqref{equa}, а также производная $u_{xt}$ непрерывны на отрезке $0 \leqslant x \leqslant l$ при $l \geqslant 0$;
	\item коэффициенты $\rho(x)$ и $k(x)$ непрерывны на отрезке $0 \leqslant x \leqslant l$.
\end{enumerate}}

Допустим, что существует два решения $u_1(x, t)$ и $u_2(x, t)$, и рассмотрим разность $v(x, t) = u_1(x, t) - u_2(x, t)$. 

Функция $v(x, t)$, очевидно, удовлетворяет однородному уравнению
\begin{equation} \label{homoeqrho}
	\rho \frac{\partial^2 v}{\partial t^2} = \frac{\partial}{\partial x} \Big(k \frac{\partial v}{\partial x}\Big)
\end{equation}
и однородным дополнительным условиям 
\begin{equation}
	\begin{rcases}
		v(x, 0) = 0, \quad v(0, t) = 0, \\
		v_t(x, 0) = 0; \quad v(l, t) = 0,
	\end{rcases}
\end{equation}
а также условию теоремы.

Докажем, что функция $v(x, t)$ тождественно равна нулю.

Рассмотрим функцию 
\begin{equation}
	E(t) = \frac{1}{2} \int \limits_{0}^{l} [k(v_x)^2 + \rho(v_t)^2] \, dx
\end{equation}
и покажем, что она не зависит от $t$. Продифференцируем $E(t)$ по $t$:
\begin{equation*}
	\frac{d E(t)}{dt} = \int \limits_{0}^{l} (k v_x v_{xt} + \rho v_t v_{tt}) \, dx.
\end{equation*}
Интегрируя по частям первое слагаемое правой части, будем иметь
\begin{equation} \label{firstint}
	\int \limits_{0}^{l} k v_x v_{xt} \, dx = [k v_x v_t]_{0}^{l} - \int \limits_{0}^{l} v_t (k v_x)_x \, dx.
\end{equation}
Подстановка обращается в нуль в силу граничных условий (из $v(0, t) = 0$ следует $v_t(0, t) = 0$ и аналогично для $x = l$). Отсюда
\begin{equation*}
	\frac{d E(t)}{dt} = \int \limits_{0}^{l} [\rho v_t v_{tt} - v_t (k v_x)_x] \, dx = \int \limits_{0}^{l} v_{t} [\rho v_{tt} - (k v_{x})_x] \, dx = 0,
\end{equation*}
т.е. $E(t) = const$. Учитывая начальные условия, получаем
\begin{equation} \label{energ}
	E(t) = const = E(0) = \frac{1}{2} \int \limits_{0}^{l} [k(v_x)^2 + \rho (v_t)^2]_{t = 0} \, dx = 0,
\end{equation}
так как 
\begin{equation*}
	v(x, 0) = 0, v_t(x, 0) = 0.
\end{equation*}

Пользуясь формулой\eqref{energ} и положительностью $k$ и $\rho$, заключаем, что 
\begin{equation*}
	v_x(x, t) \equiv 0, \quad v_t(x, t) \equiv 0,
\end{equation*}
откуда и следует тождество
\begin{equation}
	v(x, t) = const = C_0.
\end{equation}
Исходя из начального условия, находим
\begin{equation*}
	v(x, 0) = C_0 = 0;
\end{equation*}
тем самым доказано, что 
\begin{equation}
	v(x, t) \equiv 0.
\end{equation}

Следовательно, если существует две функции: $u_1(x, t)$ и $u_2(x, t)$, удовлетворяющие всем условиям теоремы, то $u_1(x, t) \equiv u_2(x, t)$.

Для второй краевой задачи функция $v = u_1 - u_2$ удовлетворяет граничным условиям 
\begin{equation}
	v_x(0, t) = 0, \quad v_x(l, t) = 0,
\end{equation}
и подстановка в формуле \eqref{firstint} также образается в нуль. Дальнейшая часть доказательства теоремы остается без изменений.

Для третьей краевой задачи доказательство требует некоторого видоизменения. Рассматривая по-прежнему два решения $u_1$ и $u_2$, получаем для их разности $v(x, t) = u_1 - u_2$ уравнение \eqref{homoeqrho} и граничные условия 
\begin{equation}
	\begin{rcases}
		v_x(0, t) - h_1 v(0, t) = 0 \quad (h_1 \geqslant 0), \\
		v_x(l, t) + h_2 v(l, t) = 0 \quad (h_2 \geqslant 0).
	\end{rcases}
\end{equation}

Представим подстановку в \eqref{firstint} в виде 
\begin{equation}
	[k v_x v_t]_{0}^{l} = - \frac{k}{2} \frac{\partial}{\partial t}[h_2 v^2(l, t) + h_1 v^2(0, t)].
\end{equation}
Интегрируя $\frac{dE}{dt}$ в пределах от нуля до $t$, получаем
\begin{equation}
	E(t) - E(0) = \int \limits_{0}^{t} \int \limits_{0}^{l} v_t[\rho v_{tt} - (k v_x)_x] \, dx \, dt - \frac{k}{2} \{h_2[v^2(l, t) - v^2(l, 0)] + h_1[v^2(0, t) - v^2(0, 0)]\},
\end{equation}
откуда в силу уравнения и начальных условий следует 
\begin{equation}
	E(t) = -\frac{k}{2} [h_2 v^2(l, t) + h_1 v^2(0, t)] \leqslant 0.
\end{equation}

Так как ввиду неотрицательности подынтегральной функции значения $E(t) \geqslant 0$, то 
\begin{equation}
	E(t) \equiv 0,
\end{equation}
а следовательно, и
\begin{equation}
	v(x, t) \equiv 0.
\end{equation}

