\subsection{Единственность решения краевой задачи (внутренней и внешней) для уравнения Лапласа.}

При исследовании стационарных процессов различной физической природы (колебания, теплопроводность, диффузия и др.) обычно приходят к уравнениям эллиптического типа. Наиболее распространенным уравнением этого типа является уравнение Лапласа
\begin{equation*}
	\Delta u = 0. 
\end{equation*}

Функция $u$ называется гармонической в области $T$, если она непрерывна в этой области вместе со своими производными до 2-го порядка и удовлетворяет уравнению Лапласа. 

Рассмотрим некоторый объем $T$, ограниченный поверхностью $\Sigma$.
Краевые задачи формулируются следующим образом.

\textit{Найти функцию $u(x, y, z)$, удовлетворяющую внутри $T$ уравнению Лапласа и граничному условию, которое может быть взято в одном из следующих видов:} 
\begin{enumerate}
	\item $u = f_1 $ на $ \Sigma $ \quad \quad \quad \quad \quad \quad \quad \quad \texttt{(первая краевая задача)},
	
	\item $\frac{\partial u}{\partial n} = f_2$ на $ \Sigma $ \quad \quad \quad \quad \quad \quad \quad \texttt{(вторая краевая задача)},
	
	\item $\frac{\partial u}{\partial n} + h(u - f_3) = 0$ на $ \Sigma $ \quad \quad \texttt{(третья краевая задача)},
\end{enumerate}
\textit{где $f_1, f_2, f_3, h$ --- заданные функции, $\partial u / \partial n$ --- производная по внешней нормали к поверхности $\Sigma$}.

Первую краевую задачу для уравнения Лапласа часто называют задачей Дирихле, вторую --- задачей Неймана.

Если ищется решение в области $T_0$, внутренней (или внешней) по отношению к поверхности $\Sigma$, то соответствующую задачу называют внутренней (или внешней) краевой задачей. 

\textit{Первая внутренняя краевая задача для уравнения Лапласа не может иметь двух различных решений}.

Допустим, что существуют две различные функции $u_1$ и $u_2$, являющиеся решениями задачи, т.е. функции, непрерывные в замкнутой области $T + \Sigma$, удовлетворяющие внутри области уравнению Лапласа и на поверхности $\Sigma$ принимающие одни и те же значения $f$. Разность этих функций $u = u_1 - u_2$ обладает следующими свойствами:
\begin{enumerate}
	\item $\Delta u = 0$ внутри области $T$;
	
	\item $u$ непрерывна в замкнутой области $T + \Sigma$;
	
	\item $u|_{\Sigma} = 0.$
\end{enumerate}

Функцию $u(M)$, таким образом, непрерывна и гармонична в области $T$ и равна нулю на границе. Как известно, всякая непрерывная функция в замкнутой области достигает своего максимального значения (п. \ref{maximum_principle}). Убедимся в том, что $u \equiv 0$. Если функция $u \not \equiv 0$ и хотя бы в одной точке $u > 0$, то она должна достигать положительного максимального значения внутри области, что невозможно. Совершенно так же доказывается, что функция $u$ не может принимать внутри $T$ отрицательных значений. Отсюда следует, что 
\begin{equation*}
	u \equiv 0.
\end{equation*}

Согласно следствию из принципа максимума, если функция $u$ гармоническая вне области $G$, и $u(\infty) = 0$, то так же можно говорить, что 
\begin{equation*}
	\abs{u(x)} \leqslant \max\limits_{\xi \in S}{\abs{u(\xi)}}, \quad x \in \bar{G_1}, G_1 = \mathbb{R}^n \ \bar{G}  
\end{equation*}

Поэтому единственность доказывается аналогично внутреннему случаю. 

\textit{Докажем, что решение второй внутренней краевой задачи определяется с точностью до произвольной постоянной.}

Доказательство проведем при дополнительном предположении, что функция $u$ имеет непрерывные первые производные в области $T + \Sigma$. 

Пусть $u_1$ и $u_2$ --- две непрерывно дифференцируемые в $T + \Sigma$ функции, удовлетворяющие уравнению $\Delta u = 0$ в $T$ и условию $\partial u/ \partial n |_{\Sigma} = f(M)$ на $\Sigma$. Для функции $u = u_1 - u_2$ будем иметь 
\begin{equation*}
	\frac{\partial u}{\partial n} \Big|_{\Sigma} = 0.
\end{equation*}
Полагая в первой формуле Грина (...%TODO) $v = u$ и учитывая соотношения $\Delta u = 0$ и $\partial u / \partial n |_{\Sigma} = 0$, получаем 
\begin{equation*}
	 \iiint \limits_{T} \Bigg[\Bigg(\frac{\partial u}{\partial x}\Bigg)^2 + \Bigg(\frac{\partial u}{\partial y}\Bigg)^2 + \Bigg(\frac{\partial u}{\partial z}\Bigg)^2\Bigg] \, d\tau = 0.
\end{equation*}

Отсюда в силу непрерывности функции $u$ и ее первых производных следует 
\begin{equation*}
	\frac{\partial u}{\partial x} = \frac{\partial u}{\partial y} = \frac{\partial u}{\partial z} \equiv 0, \quad \text{ т.е. } u \equiv const,
\end{equation*}
что и требовалось доказать. 

Изложенный метод применим и в случае неограниченной области для функций, удовлетворяющих требованиям регулярности на бесконечности.
\newline 

\textit{Докажем, что внешняя краевая задача имеет единственное решение, регулярное на бесконечности.}

В случае неограниченной области для функции, внешней к замкнутой поверхности формула Грина применима и имеет вид:
\begin{equation} \label{out_green}
	\iiint \limits_{T} u \Delta v \, d\tau = - \iiint \limits_{T} \Bigg[\frac{\partial u}{\partial x}\frac{\partial v}{\partial x} + \frac{\partial u}{\partial y}\frac{\partial v}{\partial y} + \frac{\partial u}{\partial z}\frac{\partial v}{\partial z}\Bigg] \, d\tau + \iint \limits_{\Sigma} u \frac{\partial v}{\partial n} \, d\sigma.
\end{equation}

Полагая в ней $v = u = u_1 - u_2$ и учитывая, что $\Delta u = 0$ и $\partial u / \partial n |_{\Sigma} = 0$, получаем
\begin{equation}
	\iiiint \limits_{T} (u_x^2 + u_y^2 + u_z^2) \, d\tau = 0.
\end{equation}
Отсюда в силу непрерывности производных функции $u$ следует, что 
\begin{equation*}
	u_x = 0, \quad u_y = 0, \quad u_z = 0 \text{ и } u \equiv const. 
\end{equation*}
Так как $u = 0$ на бесконечности, то
\begin{equation}
	u \equiv 0, \text{ т.е. } u_1 \equiv u_2,
\end{equation}
что и требовалось доказать.  