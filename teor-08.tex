\subsection{Распространение тепла в стержне. Постановка смешанной краевой задачи}
\subsection*{Уравнение теплопроводности в стержне}
%TODO: исправить |_{x = x_5 \\ t = t_3}
%TODO: ref типы уравнений
Простейшее уравнение параболического (см. ) типа 
\begin{equation}
  \label{eq:teplo_eq}
  u_{xx} - u_y = 0, \quad y = a^2t,
\end{equation}
то есть 
\[
  u_t = a^2u_{xx},
\]
обычно называют \emph{уравнением теплопроводности}.

Рассмотрим стержень длины $ l $, теплоизолированный с боков и 
достаточно тонкий, чтобы в любой момент времени температуру во всех 
точках поперечного сечения можно было считать одинаковой. Концы стержня будем
поддерживать при постоянных температурах $ u_1 $ и $ u_2 $. Как известно (?),
вдоль стержня тогда установится линейной распределение температуры, то есть
функция температуры имеет вид
\[
  u(x) = u_1 + \frac{u_2 - u_1}{l}x,
\]
где, например, $ u_1 > u_2 $.

Тепло\footnote{То есть энергия теплового электро-магнитного излучения.} течёт от $ u_1 $ к $ u_2 $. Экспериментальный факт: количество тепла,
протекающего через сечение\footnote{Здесь и далее имеется в виду поперечное
сечение (?).} стержня площади $ S $ за единицу времени равно 
\begin{equation}
  \label{eq:potok}
  Q = -k \frac{u_2 - u_1}{l} S = -k \frac{\partial u}{\partial x}S,
\end{equation}
где $ k $ --- коэффициент теплопроводности, зависящий от материала
стержня. Изучив конечный результат процесса, попробуем изучить сам процесс
распространения тепла в стержне.
\begin{enumerate}
  \item \textsc{Закон Фурье.} Обобщим формулу \eqref{eq:potok}. Количество
    тепла, протекающее через сечение $ x $ за промежуток
времени $ (t_1, t_2) $, равно 
\begin{equation}
  \label{eq:fourier}
  % dQ = qS\,dt,
  Q = -S\int\limits_{t_1}^{t_2}k(x) \frac{\partial u}{\partial x}(x, t)\,dt.
\end{equation}
% где 
% \[
%     q = -k(x) \frac{\partial u}{\partial x}
% \]
% --- \emph{плотность теплового потока}, то есть количество тепла, проходящего за
% единицу времени\footnote{Время бесконечно малое, а
% площадь --- нет.} через площадь в 1\,см$^2$. 
В общем случае неоднородного стержня
коэффициент $ k $ является функцией от $ x $. 
% Закон \eqref{eq:fourier} можно также записать и
% в интегральной форме.
\item Количество тепла, которое нужно сообщить стержню, чтобы повысить его
  температуру $ \Delta u(x) $, равно  
  \begin{equation}
    \label{eq:2}
    Q =\int\limits_{x_1}^{x_2} c\rho(x) S\Delta u(x)\,dx,
  \end{equation}
  где $ c $ --- удельная теплоёмкость, $ \rho(x) $ --- плотность неоднородного в
  общем случае тела. При постоянных $ \rho $, $ \Delta u $ формула упрощается до
  $ Q = cm\Delta u $.
\item Внутри стержня тоже может возникать или поглощаться тепло (например, при
  прохождении тока, химических реакций и т.\,д.). Если известна объёмная
  плотность $ F(x, t) $ внутренних тепловых источников, то выделяемое ими тепло
  на промежутке $ (x_1, x_2) $ за время $ (t_1, t_2) $
  равно 
  \begin{equation}
    \label{eq:F}
    Q = S\int\limits_{t_1}^{t_2}\int\limits_{x_1}^{x_2}F(t, x)\,dx\,dt.
  \end{equation}
\end{enumerate}
  
Применим к отрезку $ (x_1, x_2) $ и промежутку времени $ (t_1, t_2) $ закон
сохранения энергии и формулы \eqref{eq:fourier}, \eqref{eq:2}, \eqref{eq:F}, и
получим 
\[
 \int\limits_{t_1}^{t_2} \left[\left. k \frac{\partial u}{\partial
   x}(x,\tau)\right|_{x=x_2} -\left. k \frac{\partial u}{\partial x}(x,
   \tau)\right|_{x=x_1} \right] \,d\tau
   +\int\limits_{x_1}^{x_2}\int\limits_{t_1}^{t_2} F(\xi, \tau)\,d\xi\,d\tau =
     \int\limits_{x_1}^{x_2}c\rho[u(\xi, t_2) - u(\xi, t_1)]\,d\xi
\]
--- уравнение теплопроводности в интегральной форме.

Предположим, что функция
$ u(x, t) $ имеет непрерывные производные $ u_{xx} $ и $ u_t $.
Пользуясь сначала интегральной теоремой о среднем, 
\[
  \left[ \left. k \frac{\partial u}{\partial x}(x,\tau) \right|_{x=x_2} -\left.
        k \frac{\partial u}{\partial x}(x,
      \tau)\right|_{x=x_1}  \right]_{\tau=t_3}\Delta t + F(x_4, t_4) \Delta x
      \Delta t = 
      \left\{ c\rho [u(\xi, t_2) - u(\xi, t_1)] \right\}_{\xi = x_3} \Delta x,
\]
а затем дифференциальной теоремой (Лагранжа) о среднем, 
\[
    \frac{\partial}{\partial x} \left[ k \frac{\partial u}{\partial x}(x, t)
    \right]_{\substack{x = x_5 \\ t = t_3}} \Delta x \Delta t + F(x_4, t_4)\Delta x \Delta
    t = \left[ c\rho \frac{\partial u}{\partial t}(x, t)
    \right]_{\substack{x=x_3\\t=t_5}}\Delta x \Delta t,
\]
где $ t_3$, $t_4$, $t_5 $ и $ x_3 $, $ x_4 $, $ x_5 $ --- промежуточные точки
соответствующих интервалов.
Отсюда  
\[
    \frac{\partial }{\partial x} \left.\left( k \frac{\partial u}{\partial x}
        \right)\right|_{\substack{x=x_5 \\ t=t_3}} + \left.F(x,
          t)\right|_{\substack{x=x_4\\t=t_4}} =
          \left. c\rho \frac{\partial u}{\partial t}\right|_{\substack{t=t_5 \\
            x= x_3}}.
\]
Поскольку все рассуждения относились к произвольным промежуткам $ (x_1, x_2) $,
$ (t_1, t_2) $, то устремив $ x_1 $, $ x_2  \to x$ и $ t_1 $, $ t_2 \to t $,
получим уравнение  
\[
    \boxed{\frac{\partial }{\partial x} \left( k \frac{\partial u}{\partial x} \right)
    + F(x, t) = c\rho \frac{\partial u}{\partial t},}
\]
называемое \emph{уравнением теплопроводности}. 

\paragraph{Частные случаи.} Рассмотрим некоторые частные случаи.
\begin{enumerate}
  \item В случае однородного
стержня (постоянства всех коэффициентов) приходим к уравнению
\eqref{eq:teplo_eq}.

 \item Если боковые стенки проводят тепло, то согласно \emph{закону Ньютона}
количество тепла, которое потеряет стержень на единицу длины и времени, равно 
\[
    F_0 = h(u - \theta),
\]
где $ \theta(x, t) $ --- температура окружающей среды, $ h $ --- коэффициент
теплообмена. Тогда $ F(x, t) = F_1(x, t) - F_0 $ (где $ F_1 $ --- объёмная
плотность источников тепла), и в случае однородности
стержня 
\[
  u_t = a^2 u_{xx} - \alpha u + f(x, t),
\]
где $ \alpha = h/(c\rho) $, $ f(x, t) = \alpha\theta(x, t) + F_1/(c\rho) $.

\item  Коэффициенты $ k $ и $ c $ медленно зависят от температуры. Предположение об
их постоянстве обусловленно предполагаемым небольшим изменением температуры.
В ином случае уравнение станет квазилинейным.
\end{enumerate}

\subsection*{Постановка смешанной краевой задачи}
Начальное условие задаёт лишь значение функции $ u(x, t) $ в начальный момент
времени $ t_0 $.

Рассматривают три основных типа граничных условий. Любая их комбинация будет
отдельной задачей.
\begin{enumerate}
  \item На конце стержня $ x = 0 $ задана
    температура  
    \[
        u(0, t) = \mu(t)
    \]
    как функция времени.
  \item На конце\footnote{Дело в том, что, как и прежде, подразумевается, что
    тепло идёт слева направо} $ x = l $ задано значение пространственной производной 
  \[
      \frac{\partial u}{\partial x}(l, t) = \nu(t).
  \]
  К такому условию приходим, если известен торцевой тепловой поток (см. формулу
  закона Фурье
  \eqref{eq:fourier}).
  % \[
      %Q(0, t) = -k \frac{\partial u}{\partial x}(0, t).
  % \]
\item На конце $ x = l $ задано линейное соотношение между производной и функцией 
\[
  \frac{\partial u}{\partial x}(l, t) = -\lambda [u(l, t) - \theta(t)].
\]
Этот тип соответствует теплообмену по закону Ньютона. Действительно, приравнивая
плотность излучения внешнего источника (по закону Фурье) и плотность ухода тепла
(по закона Ньютона), получаем исходное граничное условие.
\end{enumerate}

Перечислим некоторые особые случаи. Пусть на конце $ x = 0 $ помещена
сосредоточенная теплоёмкость $ C_1 $ и происходит теплообмен с окружающей средой
по закону Ньютона. Такой закон теплового баланса можно записать в виде условия
\[
    C_1 \frac{\partial u}{\partial t} = k \frac{\partial u}{\partial x} -
    h(u-u_0),
\]
где $ u_0 $ --- температура внешней среды.

%TODO: неоднородная среда

Помимо перечисленных выше линейных задач, ставятся и нелинейные --- например, при
излучении торца по закону \emph{Стефана -- Больцмана} 
\[
  k \frac{\partial u}{\partial x}(0, t) = \sigma [u^4(0, t) - \theta^4(0,t)],
\]
где $ \theta $ --- температура окружающей среды, а $ \sigma $ --- постоянная
Стефана -- Больцмана.

Дополнительным условием для решения будет его непрерывность в соответствующей
замкнутой области. Функция $ u $ обязана удовлетворять уравнению лишь внутри фазового прямоугольника, то
есть при $ 0 < x < l $, $ 0 < t < T $. Требование непрерывности $ u
$ нужно для этих граничных точек, внутри она непрерывна согласно самому уравнению.

\paragraph{Предельные случаи.} В очень длинном стержне граничные условия в
течение малого промежутка времени влияют на середину стержня очень слабо.
Поэтому их можно опустить, считая стержень бесконечным.

Стержень можно считать и полубесконечным, опуская с такими же рассуждениями одно
из граничных условий, если интересующий нас участок находится близко к одному
концу и очень далеко от другого.

Опустить можно и начальные условия, если считать, что с начала распространения
тепла прошло очень много времени. Тогда считаем, что опыт длится бесконечно.
Часто, например, в подобных задачах граничные
условия периодические: $ \mu(t) = A\cos\omega t $.


