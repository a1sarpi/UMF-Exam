\subsection{Собственные значения и собственные функции задачи Штурма -- Лиувилля
в прямоугольнике. Краевые задачи для уравнений Лапласа и Пуассона в
прямоугольнике}
\subsubsection{Штурм -- Лиувилль}
%ref: Свешников, стр. 56
\paragraph{Отрезок.} Рассмотрим следующую \emph{задачу Штурма -- Лиувилля}: 
\[
  \begin{cases}
    \left( k(x) y' \right)' - q(x)y + \lambda\rho(x) y =0, &
    0<x<l,\\
    \left.\alpha_1 y' - \beta_1 y \right|_{x=0} = 0,\\
      \left. \alpha_2 y'+ \beta_2 y \right|_{x=l} = 0.
  \end{cases}
\]
Главное [линейное однородное] уравнение можно переписать в виде $ Ly +\lambda\rho y = 0 $.
Найдём ФСР $ \{y_1(x; \lambda), y_2(x; \lambda)\} $ этого уравнения. Общее
решение тогда будет иметь вид $ y(x) = C_1y_1 + C_2y_2 $, после подстановки в
граничные условия оно образует однородное СЛАУ относительно неизвестных $ C_1 $, $ C_2 $.
Критерий существования ненулевого решения этой системы известен из линейной
алгебры: 
\[
  \left| \begin{matrix}
    \alpha_1y_1'(0;\lambda) - \beta_1y_1(0; \lambda) & \alpha_1y'_2(0;\lambda) -
    \beta_1y_2(0;\lambda) \\
    \alpha_2y'_1(l; \lambda) + \beta_2y_1(l;\lambda) & \alpha_2y'_2(l;\lambda) +
    \beta_2y_2(l;\lambda)
  \end{matrix} \right| = 0.
\]
Этот критерий используется для определения собстенных значений $ \lambda $.
После этого из СЛАУ можно найти\footnote{С точностью до общего множителя, конечно,
который можно найти, если требуется нормировка функций.} соответствующие пары $ (C_1, C_2) $, тем самым
завершив решение задачи.

В ряде случаев алгоритм можно упростить. Например, если найти такую ФСР, что
граничное условие в нуле для $ y_1 $ выполняется, а для $ y_2 $ --- нет, то
после подстановки в первое граничное условие заключаем, что $ C_2 = 0 $, то есть
$ y = C_1y_1 $. Уравнение для $ \lambda $ тогда примет вид  
\[
    \alpha_2 y_1'(l; \lambda) + \beta_2 y_1(l,\lambda) = 0.
\]

Простейший случай: $ Ly = y'' $, $ \rho \equiv 1 $ с ФСР $
\{\cos\sqrt\lambda x, \sin\sqrt\lambda x\} $ при $ \lambda > 0 $; $ \{x, 1\} $
при $ \lambda = 0 $ и $ \{\ch\sqrt{-\lambda}x, \sh\sqrt{-\lambda}x\} $ при $
\lambda < 0 $.

%TODO: частные случаи или пример (?) (см. там же далее)
%TODO: свойства СЗ и СФ (?) (в приложение?)


\paragraph{Прямоугольник.}
Следующая задача \emph{Лапласа на собственные значения в прямоугольнике},  
\begin{gather*}
    \Delta u + \lambda u = 0, \quad 0 < x < a, \ 0 < y < b,\\
    \left.\alpha_1u_x - \beta_1u\right|_{x=0} = 0, \quad \left.\alpha_2u_x +
      \beta_2u\right|_{x=a} = 0,\\
      \left.\alpha_3 u_y - \beta_3u\right|_{y=0} = 0, \quad \left.\alpha_4u_y +
        \beta_4 u|_{y=b} = 0,
\end{gather*}
решается методом разделения переменных: $ u(x, y) = X(x)Y(y) $. После
подстановки и деления на $ XY $ получим 
\[
    \frac{X''}{X} = - \frac{Y''}{Y} - \lambda = -\mu.
\]
Задача распалась на две одномерные задачи Штурма -- Лиувилля: 
\[
  \begin{cases}
    X'' + \mu X = 0, \quad 0 < x < a,\\
    \alpha_1 X'(0) - \beta_1 X(0) = 0,\\
    \alpha_2 X'(a) + \beta_2 X(a) = 0;
  \end{cases} \qquad
  \begin{cases}
    Y'' + \nu Y = 0, \quad 0 < y < b,\\
    \alpha_3 Y'(0) - \beta_3 Y(0) = 0,\\
    \alpha_4 Y'(b) + \beta_4 Y(b) = 0,
  \end{cases}
\]
где $ \nu := \lambda - \mu $.

\subsubsection{Лаплас и Пуассон}
%ref: Пикулин, стр. 28; https://de.ifmo.ru/--books/0051/3/3_4/34yrlappram_1.htm
В уравнении Лапласа $ \Delta u = 0 $ с некоторыми граничными условиями
аналогично разделяем переменные и приходим к соотношению\footnote{Как правило,
  при решении нас не интересует случай $ X''/X < 0 $, однако для большей
  общности можно заменить $
\lambda^2 $ на $ \lambda $.}
\[
  \frac{X''}{X} = - \frac{Y''}{Y} =: \lambda^2,  
\]
которое порождает две одномерные задачи Штурма -- Лиувилля с соответствующими
граничными условиями. При этом первая из них --- $ Y'' + \lambda^2Y = 0 $ --- с
тригонометрической ФСР\footnote{См. выше <<простейший
случай>>.}, а вторая --- $ X'' - \lambda^2X = 0 $ --- с гиперболической (конечно,
может интересовать и линейный случай $ \lambda = 0 $).

В уравнении Пуассона $ \Delta u = f(x, y) $ с некоторыми граничными условиями
аналогично разделяем переменные и приходим к вспомогательной однородной задаче
$ X'' + \lambda^2 X = 0 $. Собственные функции $ X_n(x) $ этой задачи образуют
полную ортогональную систему, поэтому можем представить решение в виде $ u(x,
y)= \sum Y_n(y) X_n(x) $, где осталось найти лишь $ Y_n(y) $. Подставим эту
форму решения, параллельно разложив по формуле
\[
  f_n(y) = \frac{1}{\|X_n\|^2} \int\limits_{0}^{a}
  f(x, y) X_n(x)\,dx
\]
неоднородную часть $ f(x, y) $ по базису $ X_n(x) $. Получим  
\[
  \sum_{n=0}^\infty X_n''(x)Y_n(y) + \sum_{n=0}^\infty X_n(x)Y_n''(y) =
  \sum_{n=0}^\infty f_n(y)X_n(x),
\]
откуда получаем уравнение на $ Y(y) $ (с соответствующими граничными условиями)
\[
  \frac{X''_n}{X_n} Y_n + Y_n'' = f_n.
\]
Решая это НЛДУ в общем виде, получаем ответ.


