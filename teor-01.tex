\subsection{Линейные уравнения с частными производными первого порядка. Уравнения характеристик. Первый интеграл. Квазилинейные уравнения. Задача Коши.}
\label{firstorder_lineq}

%autor: Сеня

\textit{Линейным уравнением с частными производными первого порядка} называется уравнение 
\begin{equation*}
	a_1 \frac{\partial u}{\partial x_1} + \dotsc + a_n \frac{\partial u}{\partial x_n} = b, \quad a_k = a_k(x_1, \dotsc, x_n), b = b(x_1, \dotsc, x_n).
\end{equation*}

Уравнение $\dot{x} = a(x)$ называется уравнением характеристик.
\textit{Пример:} Для уравнения $\partial u / \partial x = y \partial u / \partial y$ уравнениями характеристик будут $\dot{x} = 1, \dot{y} = -y$. 

Функция $f$ называется \textit{первым интегралом} уравнения $\dot{x} = v(x)$, если ее производная по направлению поля $v$ равна нулю:
\begin{equation*}
	v_1 \frac{\partial f}{\partial x_1} + \dotsc + v_n \frac{\partial f}{\partial x_n} = 0.
\end{equation*}

Отсюда легко понять, что функция $u$ является решением однородного линейного уравнения с частными производными первого порядка, если и только если она является первым интегралом уравнения характеристик. 

\textit{Квазилинейным уравнением первого порядка называется} уравнение
\begin{equation*}
	a_1(\textbf{x}, u) \frac{\partial u}{\partial x_1} + \dotsc + a_n(\textbf{x}, u) \frac{\partial u}{\partial x_n} = b(\textbf{x}, u).
\end{equation*}

\textit{Задачей Коши} называется задача нахождения решений уравнения 
\begin{equation*}
	a_1(x_1, \dotsc, x_n, u) \frac{\partial u}{\partial x_1} + \dotsc + a_n(x_1, \dotsc, x_n, u) \frac{\partial u}{\partial x_n} = b(x_1, \dotsc, x_n, u) 
\end{equation*}
с заданной начальной функцией
\begin{equation*}
	u|_{
	\gamma} = \varphi
\end{equation*}
определяющей значения функции на поверхности $(n - 1)$-мерной.
