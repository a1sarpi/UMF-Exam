\subsection{Линейные уравнения с частными производными первого порядка. Уравнения характеристик. Первый интеграл. Квазилинейные уравнения. Задача Коши.}
\label{firstorder_lineq}

%autor: Сеня
\textsc{Определение 1.}

\textit{Уравнением в частных производных первого порядка} называется уравнение вида
\begin{equation} \label{FOlineq}
	F\Big(x_1, \dotsc, x_n, u, \frac{\partial u}{\partial x_1}, \dotsc, \frac{\partial u}{\partial x_n}\Big) = 0,
\end{equation}
где $ x_1б \dotsc, x_n $ --- независимые переменные, $ u = u(x_1, \dotsc, x_n) $ --- неизвестная функция, $ F(x_1, \dotsc, x_n, \allowbreak p_1, \dotsc, p_n) $ --- заданная непрерывно дифференцируемая функция (здесь $ p_i $ обозначают частные производные $ u'_{x_i} = \frac{\partial u}{\partial x_i}$, $i = \overline{1, n}$) в некоторой области $ G \subset \mathbb{R}^{2 n + 1}$, причем в каждой точке области G
\begin{equation*}
	\sum \limits_{i = 1}^{n}\Bigg(\frac{\partial F}{\partial p_i}\Bigg)^2 \not = 0.
\end{equation*}  

Уравнение \eqref{FOlineq} сокращенно можно записать в виде 
\begin{equation} \label{FOsimp}
	F(x, u, \nabla u) = 0, \tag{1'}
\end{equation}
где $ x = (x_1, \dotsc, x_n) $ и $\nabla u = \Big(\frac{\partial u}{\partial x_1}, \dotsc, \frac{\partial u}{\partial x_n}\Big)$.


В зависимости от того, как неизвестная функция $ u $ и ее частные производные входят в уравнение \eqref{FOlineq}, различают \textit{линейные} и \textit{нелинейные} уравнения.


\textsc{Определение 2.} \textit{(Линейные уравнения с частными производными первого порядка)}

Уравнение вида 
\begin{equation}
	a_1(x_1, \dotsc, x_n) \frac{\partial u}{\partial x_1} + \dotsc + a_n(x_1, \dotsc, x_n) \frac{\partial u}{\partial x_n} = b(x_1, \dotsc, x_n),
\end{equation}
где $ a_1, \dotsc, a_n, b \in C^1(D), D \subset \mathbb{R}^{n}$, называется \textit{линейным неоднородным уравнением с частными производными первого порядка. Если $ b(x_1, \dotsc, x_n) = 0$, то уравнение называется линейным однородным.}


\textsc{Определение 3.} \textit{(Квазилинейные уравнения)}

Уравнение вида 
\begin{equation}
	a_1(x_1, \dotsc, x_n, u) \frac{\partial u}{\partial x_1} + \dotsc + a_n(x_1, \dotsc, x_n, u) \frac{\partial u}{\partial x_n} = b(x_1, \dotsc, x_n, u),
\end{equation}
где $ a_1, \dotsc, a_n, b \in C^1(D), D \subset \mathbb{R}^{n}$, называется \textit{квазилинейным неоднородным уравнением с частными производными первого порядка. Если $ b(x_1, \dotsc, x_n, u) = 0$, то уравнение называется квазилинейным однородным.} 

\textsc{Определение 4.} \textit{(Первый интеграл)}

Первым интегралом нормальной системы
\begin{equation} \label{norm_sys}
	\begin{cases}
		\dot{x}_1 = f_1(t, x_1, \dotsc, x_n), \\
		\dotsc \\
		\dot{x}_n = f_n(t, x_1, \dotsc, x_n),
	\end{cases}
\end{equation} 
называется такая функция $ v(t, x_1, \dotsc, x_n)$, что она постоянна вдоль любого решения этой системы. Выражение $ v(t, x_1, \dotsc, x_n) = 0$ называется общим интегралом системы. 

\textit{Замечание:} Если $ v(t, x_1, \dotsc, x_n) $ --- первый интеграл системы \eqref{norm_sys}, то его производная вдоль решения равняется нулю, то есть 
\begin{equation*}
	\frac{d v}{d t} = \frac{\partial v}{\partial t} + \frac{\partial v}{\partial x_1} \dot{x}_1 + \dotsc + \frac{\partial v}{\partial x_n} \dot{x}_n = \frac{\partial v}{\partial t} + \frac{\partial v}{\partial x_1} f_1(x_1, \dotsc, x_n) + \dotsc + \frac{\partial v}{\partial x_n} f_n(x_1, \dotsc, x_n) = 0.
\end{equation*}

Справедливо и обратное, то есть функция, удовлетворяющая такому условию, является первым интегралом системы.

\textsc{Определение 5.} \textit{(Задача Коши)}

\textit{Задачей Коши} называется задача нахождения решения уравнения 
\begin{equation*}
	a_1(x, y, z) \frac{\partial z}{\partial x} + a_2(x, y, z) \frac{\partial z}{\partial y} = b(x, y, z),
\end{equation*}
проходящего через кривую
\begin{equation*}
	\begin{cases}
		x = \varphi(t), \\
		y = \psi(t), \\
		z = \eta(t).
	\end{cases}
\end{equation*}

\textsc{Определение 6.}
\textit{(Уравнения характеристик)}

Рассмотрим уравнение
\begin{equation*}
	a_1(x_1, \dotsc, x_n, u) \frac{\partial u}{\partial x_1} + \dotsc + a_n(x_1, \dotsc, x_n, u) \frac{\partial u}{\partial x_n} = 0
\end{equation*}
сопоставленная ему система обыкновенныъ дифференциальных уравнений называется \textit{системой уравнений характеристик}
\begin{equation}
	\frac{d x_1}{a_1(x_1, \dotsc, x_n)} = \frac{d x_2}{a_2(x_1, \dotsc, x_n)} = \dotsc = \frac{d x_n}{a_n(x_1, \dotsc, x_n)}.
\end{equation}

Также систему можно записать в виде:
\begin{equation}  \label{char_sys}
	\begin{cases}
		\dot{x}_1 = a_1(x_1, \dotsc, x_2), \\
		\dotsc \\
		\dot{x}_n = a_n(x_1, \dotsc, x_n).
	\end{cases} \tag{5'}
\end{equation}

\textit{Замечание}: функция $ u(x_1, \dotsc, x_n) $ является решением линейного однородного уравнения в частных проихводных первого порядка тогда и только тогда, когда является независящим от времени первым интегралом системы \eqref{char_sys}. \footnote{Все было из https://math-it.petrsu.ru/users/semenova/UMF/Lections/Urav\_Pp.pdf 
	или из https://teach-in.ru/file/methodical/pdf/diffeq1-M.pdf
}