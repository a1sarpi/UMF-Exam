\subsection{Собственные значения и собственные функции задачи Штурма -- Лиувилля в круговом секторе и в кольцевом секторе. Краевая задача для уравнения Лапласа в указанных областях}
%aut: Slava
\subsubsection{Штурм -- Лиувилль}
Решение задачи аналогично данному в \ref{sec:20} с той разницей, что задача на
$ \Phi $ будет иметь не периодические условия, а стандартные (как, например,
в решении ниже).

\subsubsection{Лаплас} Идейно отличаются от задач, разобранных в разделах \ref{sec:16} и \ref{sec:20}
%ref: Пикулин, стр. 19
лишь тем, что задача Штурма -- Лиувилля вместо периодических условий получает
стандрартные. Задача на $ R(r) $ решается так же. Граничная функция
раскладывается, вообще говоря, уже не в ряд Фурье, а по системе функций $
\Phi_n(\varphi) $.

\paragraph{Пример.} Пусть дана задача  
\label{sec:laplace_example}
\[
  \begin{cases}
    r^2u_{rr} + ru_r + u_{\varphi\varphi} = 0, & 0 < r < a, \ 0 < \varphi <
    \alpha < 2\pi,\\
    u(r, 0) = u(r, \alpha) = 0,\\
    u(a, \varphi) = A\varphi.
  \end{cases}
\]
После разделения переменных получаем соотношения\footnote{Вывод о знаке $\lambda
$ был сделан исходя из второго граничного условия.}
\[
    \frac{r^2 R'' + rR'}{R} =-\frac{\Phi''}{\Phi} =: \lambda > 0.
\]
 Решим
задачу Штурма -- Лиувилля\footnote{См. <<простейший случай>> в разделе
\ref{sec:S-L}.}
\[
  \begin{cases}
    \Phi'' + \lambda\Phi = 0,\\
    \Phi(0) = \Phi(\alpha) = 0.
  \end{cases}
\]
Тогда $ \Phi_n(\varphi) = \sin \sqrt\lambda_n\varphi $, $ \sqrt{\lambda_n} = \pi
n/\alpha$. 

Уравнение Эйлера $ r^2R'' + rR' - \lambda R = 0 $ решим подстановкой $ r^\mu $,
откуда $ \mu(\mu-1) + \mu - \lambda_n = 0 $, и $ \mu = \pm n\pi/\alpha $. С
учётом ограниченности $ R_n(r) = r^{n\pi/\alpha} $. 

Тогда  
\[
  u(r, \varphi) = \sum_{n=1}^\infty c_n r^{n\pi/\alpha}\sin \left(
  \frac{n\pi}{\alpha}\varphi \right),
\]
где коэффициент
\[
  c_na^{n\pi/\alpha} = \frac{2}{\alpha} \int\limits_{0}^{\alpha}A\varphi \sin
  \left( \frac{n\pi}{\alpha}\varphi \right) \,d\varphi \Rightarrow
  c_n = (-1)^{n+1} \frac{2\alpha A}{n\pi}
\]
был найден с помощью разложения\footnote{Рабочая формула находится по адресу
\eqref{eq:razl}.} граничной функции $ A\varphi $ в базисе $
\Phi_n(\varphi) $. Отметим, что решение имеет особенность в граничной точке $r =
a$, $\varphi = \alpha$
из-за несогласования граничных значений.
