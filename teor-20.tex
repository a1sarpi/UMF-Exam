\subsection{Собственные значения и собственные функции задачи Штурма -- Лиувилля
в круге, в круговом кольце и во внешности круга. Краевые задачи для уравнения
Лапласа в указанных областях}\label{sec:20}
\subsubsection{Штурм -- Лиувилль}
%aut: Slava
%ref: Свешников, стр. 117
Рассматривается задача
\[
  \begin{cases}
    \Delta u + \lambda u = 0,\\
    \alpha u(a, \varphi)_n + \beta u(a, \varphi) = 0.
  \end{cases}
\]
Напомним\footnote{См. раздел \ref{sec:14}.}, что  
\[
    \Delta u = \frac{1}{r} \frac{\partial}{\partial r} \left( r \frac{\partial
    u}{\partial r} \right) + \frac{1}{r^2} \frac{\partial^2 u}{\partial
  \varphi^2}.
\]
Тогда после разделения переменных получим\footnote{Знак $ \nu $ обусловлен
требуемой периодичностью $ \Phi(\varphi) $.}
\[
    \frac{r (r R')' + \lambda r^2 R}{R} = - \frac{\Phi''}{\Phi} =: \nu \geqslant
    0.
\]
Получили простейшую задачу Штурма -- Лиувилля с периодическими
условиями\footnote{См., например, раздел \ref{sec:16}.}
на $ \Phi(\varphi) $ с решением $ \Phi_n(\varphi) = A\cos n\varphi + B\sin
n\varphi $ ($ \nu_n = n^2 $) и задачу
\[
  \begin{cases}
    r^2R'' + rR' + (\lambda r^2 - n^2) R = 0,\\
    \alpha R'(a) + \beta R(a) = 0.
  \end{cases}
\]
с дополнительным условием ограниченности. Заменой $ x := r\sqrt\lambda $
уравнение
приводится к \emph{уравнению Бесселя}\footnote{См. раздел \ref{sec:bessel}.} 
\[
    x^2y'' + xy' + (x^2 - n^2) y = 0
\]
с решением  
\[
    R_n(r) = C_1 J_n(\sqrt \lambda r) + C_2 N_n (\sqrt\lambda r), \quad n = 0,
    1, 2, \ldots
\]
Здесь, возможно, потребуется исключить одну из функций, неограниченную в
заданной области. Функция $ N_n $ не ограничена в нуле, а $ J_n $ в бесконечности.
В кольцевой задаче ничего исключать не надо. Далее будем считать, что задача
внутренняя, тогда $ R_n(r) = J_n(\sqrt\lambda_n r) $ и $ \lambda_n $ ищется
численно из
граничного соотношения  
\[
    \alpha \sqrt\lambda J'_n(\sqrt\lambda a) + \beta J_n(\sqrt\lambda a) = 0.
\]
Корней может быть и больше одного, поэтому окончательно собственные функции
круга имеют вид 
\[
  u_{nk}(r, \varphi) = J_n(\sqrt{\lambda_{nk}} r)(A_{nk}\cos n\varphi +
  B_{nk}\sin
  n\varphi).
\]
Найдём скалярный квадрат этой функции $ \|u_{kn}\|^2 =
\|J_n\|^2\cdot\|\Phi_n\|^2 $, где норму $ \Phi_n $ считаем известной из
теории рядов Фурье. Найдём в цилиндрических координатах, используя
интегрирование по частям,
\[
    \|J_n\|^2 = \int\limits_{0}^{a} J^2_n(\sqrt\lambda r)r\,dr =
    \frac{1}{\lambda}\int\limits_{0}^{a\sqrt\lambda}J^2_n(x)x\,dx =
    \frac{1}{\lambda} \left[ \frac{x^2}{2}J^2_n(x) -
    \int x^2 J_n(x)J'_n(x)\,dx  \right]^{a\sqrt\lambda}_0,
\]
где из уравнения Бесселя
\[
  x^2 J_n = -x^2 J''_n - xJ'_n + n^2J_n = - x(xJ'_n)' + n^2 J_n,
\]
поэтому выражение в квадратных скобках равно
\[
  % \int J^2_n(x) x\,dx = \frac{x^2}{2}J^2_n(x) - \int x^2 J_n(x)J'_n(x)\,dx =\\=
  \frac{x^2}{2} J^2_n(x) + \int xJ_n' (xJ_n')'\,dx - n^2 \int J_n J_n' \, dx
  =
  \frac{x^2}{2} J^2_n + \frac{x^2}{2} (J'_n)^2 - \frac{n^2}{2} J^2_n.
\]
Итак, учитывая тождество $ J_n(0) = 0 $,
\[
    \|J_n\|^2 = \frac{a^2}{2} \left[ (J'_n(a\sqrt\lambda))^2 + \left( 1 -
    \frac{n^2}{a^2\lambda} \right) J_n^2 (a\sqrt\lambda) \right].
\]
Например, для задачи Дирихле ($ \alpha = 0 $, $ \beta = 1 $) имеем $
J_n(a\sqrt\lambda) = 0 $, а значит,  
\[
  \|J_n\|^2 = \frac{a^2}{2} (J'_n(a\sqrt{\lambda_{nk}})^2.
\]





\subsubsection{Лаплас}
%ref: Свешников, стр. 160
\paragraph{Дирихле.} Задача Дирихле для круга была разобрана в разделе \ref{sec:16}. Задача Дирихле
для кругового кольца отличается лишь тем, что из ФСР $ \{1, \ln r, r^n, r^{-n}\}
$, полученной в ходе решения уравнения Эйлера на $ R(r) $, не стоит исключать ни
одной функции:
\[
  u(r,\varphi) = a_0 + b_0\ln r + \sum_{n=1}^\infty [(a_nr^n + b_nr^{-n})\cos
  n\varphi + (c_n r^n + d_n r^{-n})\sin n\varphi].
\]
Хода решения это никак не изменит, однако теперь потребуется
разложить в ряд Фурье две граничных функции (для $ R_1 $ и $ R_2 $) и приравнять
соответствующие коэффициенты.

\paragraph{Нейман.} 
%ref: Пикулин, стр. 25
Задача Неймана тоже мало чем отличается от разобранной в \ref{sec:16}. В случае задачи внутри круга радиуса $ R $ с центром
в начале координат внешняя нормальная производная есть $ u_{\mathbf n}|_{r = a}
= u_r|_{r = a}$. Общее решение внутренней и внешней задачи такое же, как в
\ref{sec:16}. Изменятся только соотношения между коэффициентами. Так, после
взятия производной теперь для
внутренней задачи $
A_n = (\alpha_n)/(na^{n-1}) $, $ B_n = (R\beta_n)/(na^{n-1}) $. Коэффициент $
\alpha_0 $ в данном случае быть ненулевым не может. Для внешней задачи $
u_{\mathbf n}|_{r=a} = -u_r|_{r=a} $, остальное аналогично.

Для смешанных краевых задач ход решения аналогичный.
