\subsection{Собственные значения и собственные функции задачи Штурма -- Лиувилля
в круге, в круговом кольце и во внешности круга. Краевые задачи для уравнения
Лапласа в указанных областях}\label{sec:20}
\subsubsection{Штурм -- Лиувилль}
%ref: Свешников, стр. 60 (?)
Вопрос был полностью разобран в разделе \ref{sec:16}. Во всех указанных случаях
получается задача Штурма -- Лиувилля с периодическими граничными условиями.

\subsubsection{Лаплас}
%ref: Свешников, стр. 160
\paragraph{Дирихле.} Задача Дирихле для круга была разобрана в разделе \ref{sec:16}. Задача Дирихле
для кругового кольца отличается лишь тем, что из ФСР $ \{1, \ln r, r^n, r^{-n}\}
$, полученной в ходе решения уравнения Эйлера на $ R(r) $, не стоит исключать ни
одной функции:
\[
  u(r,\varphi) = a_0 + b_0\ln r + \sum_{n=1}^\infty [(a_nr^n + b_nr^{-n})\cos
  n\varphi + (c_n r^n + d_n r^{-n})\sin n\varphi].
\]
Хода решения это никак не изменит, однако теперь потребуется
разложить в ряд Фурье две граничных функции (для $ R_1 $ и $ R_2 $) и приравнять
соответствующие коэффициенты.

\paragraph{Нейман.} 
%ref: Пикулин, стр. 25
Задача Неймана тоже мало чем отличается. В случае круга радиуса $ R $ с центром
в начале координат внешняя нормальная производная есть $ u_{\mathbf n}|_{r = a}
= u_r|_{r = a}$. Решение внутренней и внешней задачи такое же, как в
\ref{sec:16}. Изменятся только соотношения между коэффициентами. Так, теперь для
внутренней задачи $
A_n = (\alpha_n)/(na^{n-1}) $, $ B_n = (R\beta_n)/(na^{n-1}) $. Коэффициент $
\alpha_0 $ в данном случае быть ненулевым не может. Для внешней задачи $
u_{\mathbf n}|_{r=a} = -u_r|_{r=a} $, остальное аналогично.

