\subsection{Собственные значения и собственные функции задачи Штурма -- Лиувилля
	в круге, в круговом кольце и во внешности круга. Краевые задачи для уравнения
	Лапласа в указанных областях}\label{sec:20}
\subsubsection{Штурм -- Лиувилль}
%aut: Slava
%ref: Свешников, стр. 117
\paragraph{Внутренняя задача (круг и кольцо).}
Рассматривается задача\footnote{В кольцевой задаче добавится, разумеется, ещё
	одно граничное условие, что, впрочем, не повлияет на ход решения.}
\[
\begin{cases}
	\Delta u + \lambda u = 0,\\
	\alpha \frac{\partial u(a, \varphi)}{\partial \mathbf n} + \beta u(a, \varphi) = 0.
\end{cases}
\]
Из однородности условий заключаем, что $ \lambda \geqslant 0 $. Иные случаи будут разобраны в разделе
\ref{sec:laplace_cylinder}. Напомним\footnote{См. раздел \ref{sec:14}.}, что  
\[
\Delta u = \frac{1}{r} \frac{\partial}{\partial r} \left( r \frac{\partial
	u}{\partial r} \right) + \frac{1}{r^2} \frac{\partial^2 u}{\partial
	\varphi^2}.
\]
Тогда после разделения переменных получим\footnote{Знак $ \nu $ обусловлен
	требуемой периодичностью $ \Phi(\varphi) $.}
\[
\frac{r (r R')' + \lambda r^2 R}{R} = - \frac{\Phi''}{\Phi} =: \nu \geqslant
0.
\]
Получили простейшую задачу Штурма -- Лиувилля с периодическими
условиями\footnote{См., например, раздел \ref{sec:16}.}
на $ \Phi(\varphi) $ с решением $ \Phi_n(\varphi) = A\cos n\varphi + B\sin
n\varphi $ ($ \nu_n = n^2 $) и задачу
\begin{equation}
	\label{eq:bessel_eq}
	\begin{cases}
		r^2R'' + rR' + (\lambda r^2 - n^2) R = 0,\\
		\alpha R'(a) + \beta R(a) = 0.
	\end{cases}
\end{equation}
с дополнительным условием ограниченности. При $ \lambda = 0 $ полученное
уравнение вырождается в уравнение Эйлера с ФСР $ \{1, \ln r , r^n, r^{-n}\}$, $n
\in \mathbb N$. При $ \lambda > 0 $ заменой $ x := r\sqrt\lambda $
уравнение
приводится к \emph{уравнению Бесселя}\footnote{См. раздел \ref{sec:bessel}.} 
\[
x^2y'' + xy' + (x^2 - n^2) y = 0
\]
с решениями $ \{J_n(\sqrt \lambda r), N_n(\sqrt\lambda r)\} $, $ n = 0, 1, \ldots $
% \[
%     R_n(r) = C_1 J_n(\sqrt \lambda r) + C_2 N_n (\sqrt\lambda r), \quad n = 0,
%     1, 2, \ldots
% \]
Здесь, возможно, потребуется исключить функции, неограниченные в
заданной области. Функция $ N_n $ неограниченна в нуле, а $ J_n $ ограничена
везде.
В кольцевой задаче ничего исключать не надо. Далее будем считать, что задача
круговая внутренняя, тогда $ R_n(r) = r^n $ при $ \lambda = 0 $, $ R_n(r) =
J_n(\sqrt\lambda_n r) $ при $ \lambda > 0 $, и положительные $
\lambda_n $ ищутся
численно из
граничного соотношения  
\[
\alpha \sqrt\lambda J'_n(\sqrt\lambda a) + \beta J_n(\sqrt\lambda a) = 0.
\]
Положительных корней при классических условиях счётное число, поэтому окончательно собственные функции
круга имеют вид 
\[
u_{nk}(r, \varphi) = \{r^n; \ J_n(\sqrt{\lambda_{nk}} r)\}(A_{nk}\cos n\varphi +
B_{nk}\sin
n\varphi).
\]
%TODO: проверить запись, суммирование
При этом при данных условиях $ r^n $ всегда исключается из
рассмотрения\footnote{Поскольку невыполнимо условие Неймана $ R'(a) = 0 $, а $
	\lambda = 0 $ в остальных случаях не даёт нетривиальных решений.}. Найдём скалярный квадрат этой функции $ \|u_{kn}\|^2 =
\|J_n\|^2\cdot\|\Phi_n\|^2 $, где норму $ \Phi_n $ считаем известной из
теории рядов Фурье. Найдём в цилиндрических координатах, используя
интегрирование по частям,
\[
\|J_n\|^2 = \int\limits_{0}^{a} J^2_n(\sqrt\lambda r)r\,dr =
\frac{1}{\lambda}\int\limits_{0}^{a\sqrt\lambda}J^2_n(x)x\,dx =
\frac{1}{\lambda} \left[ \frac{x^2}{2}J^2_n(x) -
\int x^2 J_n(x)J'_n(x)\,dx  \right]^{a\sqrt\lambda}_0,
\]
где из уравнения Бесселя
\[
x^2 J_n = -x^2 J''_n - xJ'_n + n^2J_n = - x(xJ'_n)' + n^2 J_n,
\]
поэтому выражение в квадратных скобках равно
\[
% \int J^2_n(x) x\,dx = \frac{x^2}{2}J^2_n(x) - \int x^2 J_n(x)J'_n(x)\,dx =\\=
\frac{x^2}{2} J^2_n(x) + \int xJ_n' (xJ_n')'\,dx - n^2 \int J_n J_n' \, dx
=
\frac{x^2}{2} J^2_n + \frac{x^2}{2} (J'_n)^2 - \frac{n^2}{2} J^2_n.
\]
Итак, учитывая тождество $ J_n(0) = 0 $, для $ r = a $ получим
\[
\|J_n\|^2 = \frac{a^2}{2} \left[ (J'_n(a\sqrt\lambda))^2 + \left( 1 -
\frac{n^2}{a^2\lambda} \right) J_n^2 (a\sqrt\lambda) \right].
\]
Например, для задачи Дирихле ($ \alpha = 0 $, $ \beta = 1 $) имеем $
J_n(a\sqrt\lambda) = 0 $, а значит,  
\[
\|J_n\|^2 = \frac{a^2}{2} (J'_n(a\sqrt{\lambda_{nk}}))^2.
\]
Для задачи Неймана ($ \alpha = 1 $, $ \beta = 0 $), обратно, $ J'_n(a\sqrt\lambda) = 0. $

\paragraph{Внешняя круговая задача (задача дифракции).}
%ref: Боголюбов, стр. 293, Пикулин, стр. 52
Чтобы выделить единственное решение, на бесконечности нужно поставить
дополнительное условие --- условие излучения, выделяющее уходящую волну. На
плоскости условие имеет один из двух видов 
\[
u_r \pm i\sqrt\lambda u = o \left( \frac{1}{\sqrt r} \right), \quad
r\to\infty.
\]
Будем для примера считать, что выбран знак <<минус>>. Представим решение для $ R(r) $, учитывая указанную асимптотику, в виде
\emph{функций Ханкеля первого рода} 
\[
R_{nk}(r) = A H^{(1)}_n(\sqrt{\lambda_{nk} r),
\]
% а решение имеет вид 
% \[
%   u = \sum_{n=0}^\infty \frac{H_n^{(1)}(\sqrt\lambda_{nk}
	%     r)}{\alpha\sqrt\lambda_{nk}
	%     H^{(1)}'_n(\sqrt\lambda_{nk} a) - \beta H_n^{(1)}(\sqrt\lambda_{nk} a)} (\alpha_n\cos
%   n\varphi + \beta_n \sin n\vqrphi),
% \]
% где $ \alpha $, $ \beta $ --- коэффициенты разложения в ряд Фурье граничной
% функции.

%TODO: пояснить. Исправить для однородных граничных условий

Краевая задача с условием излучения со знаком <<плюс>> для $ R(r) $ имеет
решения  
\[
R_{nk}(r) = AH^{(2)}_n(\sqrt{\lambda_{nk}} r).
\]

% \[
%   u = \alpha_0 \frac{H_0^{(2)}(\sqrt\lambda)}{H_0^{(2)}(\sqrt\lambda a)} +
%   \sum_{n=1}^\infty \frac{H_n^{(2)}(\sqrt\lambda r)}{H_n^{(2)}(\sqrt\lambda)}
%   (\alpha_n \cos n\varphi + \beta_n\sin n\varphi),
% \]
% где $ \alpha $, $ \beta $ --- коэффициенты разложения в ряд Фурье граничной
% функции.



\subsubsection{Лаплас}
%ref: Свешников, стр. 160
\paragraph{Дирихле.} Задача Дирихле для круга была разобрана в разделе \ref{sec:16}. Задача Дирихле
для кругового кольца отличается лишь тем, что из ФСР $ \{1, \ln r, r^n, r^{-n}\}
$, полученной в ходе решения уравнения Эйлера на $ R(r) $, не стоит исключать ни
одной функции:
\[
u(r,\varphi) = a_0 + b_0\ln r + \sum_{n=1}^\infty [(a_nr^n + b_nr^{-n})\cos
n\varphi + (c_n r^n + d_n r^{-n})\sin n\varphi].
\]
Хода решения это никак не изменит, однако теперь потребуется
разложить в ряд Фурье две граничных функции (для $ R_1 $ и $ R_2 $) и приравнять
соответствующие коэффициенты.

\paragraph{Нейман.} 
%ref: Пикулин, стр. 25
Задача Неймана тоже мало чем отличается от разобранной в \ref{sec:16}. В случае задачи внутри круга радиуса $ R $ с центром
в начале координат внешняя нормальная производная есть $ u_{\mathbf n}|_{r = a}
= u_r|_{r = a}$. Общее решение внутренней и внешней задачи такое же, как в
\ref{sec:16}. Изменятся только соотношения между коэффициентами. Так, после
взятия производной теперь для
внутренней задачи $
A_n = (\alpha_n)/(na^{n-1}) $, $ B_n = (R\beta_n)/(na^{n-1}) $. Коэффициент $
A_0 $ в данном случае будет произвольным. Для внешней задачи $
u_{\mathbf n}|_{r=a} = -u_r|_{r=a} $, остальное аналогично.

Для смешанных краевых задач ход решения аналогичный.
