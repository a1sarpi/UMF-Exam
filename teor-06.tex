\subsection{Метод разделения переменных для уравнения колебаний на отрезке.}

Метод разделения переменных, или метод Фурье является одним из наиболее распространенных методов решения уравнений с частными производными.

Изложение этого метода проведем для задачи колебания струны, закрепленной на концах (т.е. на отрезке). 

Итак, будем искать решение уравнения 
\begin{equation} \label{oscil}
	u_{tt} = a^2 u_{xx},
\end{equation}
удовлетворяющее однородным граничным условиям 
\begin{equation} \label{bord_cond}
	u(0, t) = 0, u(l, t) = 0
\end{equation}
и начальным условиям 
\begin{equation} \label{init_cond}
	\begin{cases}
		u(x, 0) = \varphi(x),
		\\
		u_t(x, 0) = \psi(x).
	\end{cases}
\end{equation}

Уравнение \eqref{oscil} линейно и однородно, поэтому сумма частных решений также является решением этого уравнения. Имея достаточно большое число частных решений, можно попытаться при помощи суммирования их с некоторыми коэффициентами найти искомое решение. 

Поставим основную вспомогательную задачу.

\textit{Найти решение уравнения}
\begin{equation*}
	u_{tt} = a^2 u_{xx},
\end{equation*}
\textit{не равное тождественно нулю, удовлетворяющее однородным граничным условиям}
\begin{equation}
	\label{ucond}
	\begin{cases}
		u(0, t) = 0,
		\\
		u(l, t) = 0
	\end{cases}
\end{equation}
\textit{и представимое в виде произведения}
\begin{equation} \label{sol_form}
	u(x, t) = X(x) T(t),
\end{equation}
\textit{где $X(x)$ --- функция только переменного $x$, $T(t)$ --- функция только переменного $t$.}

Подставляя предполагаемую форму решения \eqref{sol_form} в уравнение \eqref{oscil}, получим 
\begin{equation*}
	X'' T = \frac{1}{a^2} T'' X,
\end{equation*}
или, после деления на $X T$,
\begin{equation} \label{need}
	\frac{X''(x)}{X(x)} = \frac{1}{a^2} \frac{T''(t)}{T(t)}.
\end{equation}

Чтобы функция \eqref{sol_form} была решением уравнения \eqref{oscil}, равенство \eqref{need} должно удовлетворяться тождественно, т.е. для всех значений назависимых переменных $0 < x < l$, $t > 0$. Правая часть равенства \eqref{need} является функцией только переменного $t$, а левая --- только $x$. Фиксируя, например, некоторое значение $x$ и меняя $t$ (или наоборот), получим, что правая и левая части равенства \eqref{need} при изменении своих аргементов сохраняют постоянное значение
\begin{equation} \label{comstlambd}
	\frac{X''(x)}{X(x)} = \frac{1}{a^2} \frac{T''(t)}{T(t)} = -\lambda,
\end{equation}
где $\lambda$ --- постоянная, которую для удобства последующих выкладок берем со знаком "минус", ничего не предполагая при этом о ее знаке.

Из соотношения \eqref{comstlambd} получаем обыкновенные дифференциальные уравнения для определения функций $X(x)$ и $T(t)$:
\begin{align}
	&X''(x) + \lambda X(x) = 0, \quad X(x) \not \equiv 0, \\
	&T''(t) + a^2 \lambda T(t) = 0, \quad T(t) \not \equiv 0. \label{Teq}
\end{align}

Граничные условия \eqref{bord_cond} дают
\begin{align*}
	u(0, t) = X(0) T(t) = 0, \\
	u(l, t) = X(l) T(t) = 0.
\end{align*}
Отсюда следует, что функция $X(x)$ должна удовлетворять дополнительным условиям 
\begin{equation}
	X(0) = X(l) = 0,
\end{equation}
так как иначе мы имели бы
\begin{equation*}
	T(t) \equiv 0 \text{ и } u(x, t) \equiv 0,
\end{equation*}
в то время как задача состоит в нахождении нетривиального решения. Для функции $T(t)$ в основной вспомогательной задаче никаких дополнительных условий нет. 

Таким образом, в связи с нахождением функции $X(x)$ мы приходим к простейшей задаче о собственных значениях.
\begin{equation} \label{euigen}
	\begin{cases} 
		X''(x) + \lambda X(x) = 0, \\
		X(0) = X(l) = 0.
	\end{cases}
\end{equation}

Рассмотрим отдельно возможные значения параметра $\lambda$:
\begin{enumerate}
	\item $\lambda < 0$
	\begin{equation*}
		X(x) = C_1 e^{\sqrt{- \lambda} x} + C_2 e^{-\sqrt{- \lambda} x}.
	\end{equation*}
	Граничные условия дают
	\begin{align*}
		X(0) = C_1 + C_2 = 0, \\
		X(l) = C_1 e^{\alpha} + C_2 e^{-\alpha} = 0 \quad (\alpha = l\sqrt{-\lambda}),
	\end{align*}
	т.е. 
	\begin{equation*}
		C_1 = - C_2 \text{ и } C_1(e^{\alpha} + e^{-\alpha}) = 0.
	\end{equation*}
	Но в рассматриваемом случае $\alpha$ действительно и положительно, так что $e^{\alpha} - e^{-\alpha} \not = 0$. Поэтому 
	\begin{equation*}
		C_1 = C_2 = 0,
	\end{equation*}
	следовательно, 
	\begin{equation*}
		X(x) \equiv 0.
	\end{equation*}
	
	\item $\lambda = 0$
	\begin{equation*}
		X(x) = C_1 x + C_2.
	\end{equation*}
	Граничные условия дают
	\begin{align*}
		X(0) = [C_1 x + C_2]_{x = 0} = C_2 = &~0, \\
		X(l) = C_1 l = &~0,
	\end{align*}
	т.е. $C_1 = 0$ и $C_2 = 0$ и, следовательно, 
	\begin{equation*}
		X(x) \equiv 0. 
	\end{equation*}
	
	\item $\lambda > 0$
	\begin{equation*}
		X(x) = D_1 \cos{\sqrt{\lambda} x} + D_2 \sin{\sqrt{\lambda} x}.
	\end{equation*}
	Граничные условия дают
	\begin{align*}
		X(0) = D_1 = 0, \\
		X(l) = D_2 \sin{\sqrt{\lambda}l} = 0. 
	\end{align*}
	Если $X(x)$ не равно тождественно нулю, то $D_2 \not = 0$, поэтому 
	\begin{equation*}
		\sin{\sqrt{\lambda}l} = 0,
	\end{equation*}
	или
	\begin{equation*}
		\sqrt{\lambda} = \frac{\pi n}{l},
	\end{equation*}
	где $n$ --- любое целое число. Следоовательно, нетривиальные решения задачи \eqref{euigen} возможны лишь при значениях 
	\begin{equation*}
		\lambda = \lambda_n = \Big(\frac{\pi n}{l}\Big)^2.
	\end{equation*}
	Этим собственным значениям соответствуют собственные функции \begin{equation*}
		X_n(x) = D_n \sin{\frac{\pi n}{l} x},
	\end{equation*}
	где $D_n$ --- произвольная постоянная.
	
	
	Итак, только при значениях $\lambda$, равных
	\begin{equation*}
		\lambda_n = \Big(\frac{\pi n}{l}\Big)^2.
	\end{equation*}
	существуют нетривиальные решения задачи \eqref{euigen}
	\begin{equation*}
		X_n(x) = \sin{\frac{\pi n}{l} x},
	\end{equation*}
	определяемые с точностью до произвольного множителя, который мы положили равным единице. Этим же значениям $\lambda_n$ соответствуют решения уравнения \eqref{Teq}
	\begin{equation*}
		T_n(t) = A_n \cos{\frac{\pi n}{l} a t} + B_n \sin{\frac{\pi n}{l} a t},
	\end{equation*}
	где $A_n$ и $B_n$ --- произвольные постоянные.
\end{enumerate}
	
Возвращаясь к задаче \eqref{oscil}---\eqref{init_cond} заключаем, что функции 
	\begin{equation*}
		u_n(x, t) = X_n(x) T_n(t) = \Big(A_n \cos{\frac{\pi n}{l} a t} + B_n \sin{\frac{\pi n}{l} a t}\Big) \sin{\frac{\pi n}{l} x}
	\end{equation*}
	являются частными решениями уравнения \eqref{oscil}, удовлетворяющими граничными условиям \eqref{ucond} и представивым в виде произведения \eqref{sol_form}. Эти решения могут удовлетворить начальным условиям \eqref{init_cond} нашей задачи только для частных случаев начальных функций $\varphi(x)$ и $\psi(x)$.
	
	Обратимся к решению задачи \eqref{oscil}---\eqref{init_cond} в общем случае 
	\begin{equation} \label{sol}
		u(x, t) = \sum \limits_{n = 1}^{\infty} u_n(x, t) = \sum \limits_{n = 1}^{\infty} \Big(A_n \cos{\frac{\pi n}{l} a t} + B_n \sin{\frac{\pi n}{l} a t}\Big) \sin{\frac{\pi n}{l} x}
	\end{equation} 
	В силу линейности оно также удовлетворяет граничным условиям. Начальные условия позволят определить коэффициенты $A_n$ и $B_n$.
	\begin{equation} \label{formulas}
		\begin{cases}
			u(x, 0) = \varphi(x) = \sum \limits_{n = 1}^{\infty} u_n(x, 0) = \sum \limits_{n = 1}^{\infty} A_n \sin{\frac{\pi n}{l} x}, \\
			u_t(x, 0) = \psi(x) = \sum \limits_{n = 1}^{\infty} \frac{\partial u_n}{\partial t}(x, 0) = \sum \limits_{n = 1}^{\infty} \frac{\pi n}{l} a B_n \sin{\frac{\pi n}{l} x}.
		\end{cases}
	\end{equation}
	Из теории рядов Фурье известно, что произвольная кусочно-непрерывная и кусочно-дифференцируемая функция $f(x)$, заданная в промежутке $0 \leqslant x \leqslant l$, разлагается в ряд Фурье
	\begin{equation}
		f(x) = \sum \limits_{n = 1}^{\infty} b_n \sin{\frac{\pi n}{l} x},
	\end{equation}
	где
	\begin{equation}
		b_n = \frac{2}{l} \int \limits_{0}^{l} f(\xi) \sin{\frac{\pi n}{l}\xi} \, d\xi. 
	\end{equation}
	Если функции $\varphi(x)$ и $\psi(x)$ удовлетворяют условиям разложения в ряд Фурье, то
	\begin{align}
		\varphi(x) = \sum \limits_{n = 1}^{\infty} \varphi_{n} \sin{\frac{\pi n}{l} x}, \quad \varphi_n = \frac{2}{l} \int \limits_{0}^{l} \varphi(\xi) \sin{\frac{\pi n}{l} \xi} \, d\xi, \\
		\psi(x) = \sum \limits_{n = 1}^{\infty} \psi_{n} \sin{\frac{\pi n}{l} x}, \quad \psi_n = \frac{2}{l} \int \limits_{0}^{l} \psi(\xi) \sin{\frac{\pi n}{l} \xi} \, d\xi.
	\end{align}
	
	Сравнение этих рядов с формулами \eqref{formulas} показывает, что для выполнения начальных условий надо положить 
	\begin{equation}
		A_n = \varphi_n, \quad B_n = \frac{l}{\pi n a} \psi_n,
	\end{equation}
	чем полностью определяется функция \eqref{sol}, лающая решение исследуемой задачи. 