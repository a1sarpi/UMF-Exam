\subsection{Регуляризация степенных особенностей. Сингулярная обобщённая функция $\mathcal{P} \frac{1}{x}$. Формула Сохоцкого.}


\paragraph{Регуляризация степенных особенностей.} Регуляризацией функции $f(x)$, имеющей неинтегрируемую особенность вида $\frac{1}{(x - x_0)^{\alpha}}$, называется функционал $(f, \varphi)$, который для пробных функций $\varphi(x)$, равных нулю в окресности точки $x_0$, выражается интегралом $\int f(x) \varphi(x) \, dx$.

\paragraph{Сингулярные обобщенные функции.}

В соответствии с определение, данным в предыдущем пунке, сингулярную функцию нельзя отождествить ни с какой локально интегрируемой функцией. Простейшим примером сингулярной обобщенной функции является $\delta$-функция Дирака.

\begin{equation}
	(\delta, \varphi) = \varphi(0), \quad \varphi \in \mathcal{D}.
\end{equation}
Очевидно, $\delta \in \mathcal{D}', \delta(x) = 0, x \not = 0$, так что $\operatorname{spt}{\delta} = \{0\}$ ($\operatorname{spt}$ --- это носитель, от слова \textit{support}).
 



