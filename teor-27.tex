\subsection{Регуляризация степенных особенностей. Сингулярная обобщённая функция $\mathcal{P} \frac{1}{x}$. Формула Сохоцкого.}

\paragraph{Главное значение интеграла по Коши}

ОПРЕДЕЛЕНИЕ. Для особой точки $\infty$. Пусть $f(x)$ определена на интервале $(-\infty, +\infty)$
и $\forall A : \left| \int\limits_{-A}^A f(x) \, dx \right| < \infty$, но несобственный интеграл 
$\int\limits_{-\infty}^{+\infty} f(x) \, dx$ расходится, то
\[
  v.p. \int\limits_{-\infty}^{+\infty} f(x) \, dx = \lim_{A \to +\infty} \int\limits_{-A}^A f(x) \, dx.
\]
если соответствующий предел существует\footnote{\url{https://en.wikipedia.org/wiki/Cauchy_principal_value}, на русском тоже нормально написано}.

Например, для интеграла $\int_{\mathbb{R}} x \, dx$ главное значение будет 0.

ОПРЕДЕЛЕНИЕ. Для конечной особой точки.
\[
  v.p. \int\limits_a^b f(x) \, dx = \lim_{\varepsilon \to +0} \left( 
    \int\limits_a^{c-\varepsilon} f(x) \, dx + \int\limits_{c+\varepsilon}^b f(x) \, dx \right)
    = \lim_{\varepsilon\to+0} \left( \int\limits_{a}^{c-\varepsilon} + \int\limits_{c+\varepsilon}^b \right) f(x) \, dx,
\]
если соответственно, существует такой предел, а соответствующие интегралы сходятся при $\varepsilon \in U_\delta (0)$.

\paragraph{Регуляризация степенных особенностей.}
Всякий функционал f называется регуляризацией функции g, если для любых основных функций $\varphi$,
равных нулю в окрестности особой точки g, действует по формуле $(f, \varphi) = \int g(x) \varphi(x) \, dx$.

Как сказал Покровский, регуляризация -- это придание смысла тому, что смысла не имеет.

В случае степенных особенностей, регуляризация это <<вырезание>> некоторой окрестности особой точки.
Если на всём пространстве кроме этой окрестности всё, очевидно из определения, однозначно, то
происходящее в этой окрестности можно задать огромным числом способов. Основной метод этого вырезания
состоит в использованиии интеграла в смысле главного значения. Регуляризации, построенные таким
образом обозначаются $\mathcal{P} f$:
\[
  \left( \mathcal{P} \dfrac{1}{|x-x_0|^m}, \varphi \right) = v.p. \int \dfrac{\varphi(x)}{|x-x_0|^m} \, dx.
\]
Впринцепе этой записи уже достаточно, чтобы понять что это такое и использовать в каких-то задачах,
однако каждый раз вычислять предел от интеграла, стоящий в определении главного значения,
неудобно и бессмысленно.


\paragraph{Сингулярная обобщенная функция $\mathcal{P}{\frac{1}{x}}$}

В соответствии с определением регулярной обобщенной функции, сингулярную функцию нельзя отождествить ни с какой локально интегрируемой функцией. Простейшим примером такой функции является $\delta$-функция Дирака.

Введем линейный функционал $\mathcal{P}{\frac{1}{x}}$, действующий по формуле
\begin{equation*}
	(\mathcal{P}{\frac{1}{x}}, \varphi)
  \equiv v.p. \int \frac{\varphi(x)}{x} \, dx
  = \lim \limits_{\varepsilon \to +0} \Big(\int_{-\infty}^{-\varepsilon} + \int_{+\varepsilon}^{+\infty}\Big) \frac{\varphi(x)}{x} \, dx, \quad \varphi \in \mathcal{D}(\mathbb{R}^1).
\end{equation*}
впринцепе уже этого достаточно, но каждый раз считать этот предел будет не очень удобно, поэтому
можно придумать как дальше расписать этот предел чтобы всё было замечательно.

\begin{multline*}
  (\mathcal{P} \dfrac{1}{x}, \varphi)
  = \lim_{\varepsilon\to+0} \left(\int_{-\infty}^{-\varepsilon} + \int_{\varepsilon}^{+\infty}\right)
    \dfrac{\varphi(x) - \varphi(0) + \varphi(0)}{x} \, dx
    = \lim_{\varepsilon\to+0} \left[ \int_{-\infty}^{-\varepsilon}  \right] 
\end{multline*}

\paragraph{Формула Сохоцкого.}\footnote{Источник: Владимиров В.С. стр. 77}

Установим равенство
\begin{equation}
	\label{huinya_s_potolka}
	\lim \limits_{\varepsilon \to +0} \int \frac{\varphi(x)}{x + i \varepsilon} \, dx = - i \pi \varphi(0) + \text{V.p.} \int \frac{\varphi(x)}{x} \, dx, \quad \varphi \in \mathcal{D}.
\end{equation}

Действительно, если $\varphi(x) = 0$ при $\abs{x} > R$, то 
\begin{align*}
	\lim \limits_{\varepsilon \to +0} \int \frac{\varphi(x)}{x + i \varepsilon} \, dx = \lim \limits_{\varepsilon \to +0} \int_{-R}^{R} \frac{x - i \varepsilon}{x^2 + \varepsilon^2} \varphi(x) \, dx = \\
	= \varphi(0) \lim \limits_{\varepsilon \to +0} \int_{-R}^{R} \frac{x - i \varepsilon}{x^2 + \varepsilon^2} \, dx + \lim \limits_{\varepsilon \to +0} \int_{-R}^{R} \frac{x - i \varepsilon}{x^2 + \varepsilon^2} [\varphi(x) - \varphi(0)] \, dx = \\
	= - 2 i \varphi(0) \lim \limits_{\varepsilon \to +0} \arctan{\frac{R}{\varepsilon}} + \int_{-R}^{R} \frac{\varphi(x) - \varphi(0)}{x} \, dx = \\
	= - i \pi \varphi(0) + \text{V.p.} \int \frac{\varphi(x)}{x} \, dx.
\end{align*}

Соотношение \eqref{huinya_s_potolka} означает, что существует предел последовательности $1/(x + i \varepsilon)$ в $\mathcal{D}$, $\varepsilon \to +0$, который мы обозначим $1/(x + i0)$, и этот предел равен $-i \pi \delta(x) + \mathcal{P}{\frac{1}{x}}$. Итак,
\begin{equation}
	\label{sohoc}
	\frac{1}{x + i 0} = - i \pi \delta(x) + \mathcal{P}{\frac{1}{x}}.
\end{equation}
Аналогично, 
\begin{equation}
	\label{sohoc'}
	\frac{1}{x - i0} = i \pi \delta(x) + \mathcal{P}{\frac{1}{x}}. 
\end{equation}

Формулы \eqref{sohoc} и \eqref{sohoc'} называются формулами Сохоцкого. Они широко используются, например, в квантовой физике.



