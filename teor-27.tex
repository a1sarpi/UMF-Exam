\subsection{Регуляризация степенных особенностей. Сингулярная обобщённая функция $\mathcal{P} \frac{1}{x}$. Формула Сохоцкого.}
%autor: Сеня (изначально), но вообще скорее Саша

\paragraph{Регуляризация степенных особенностей.}\footnote{Источник: https://fizik.ilyam.org/files/konspekts/matfizika6-distributions2.pdf стр. 13} Регуляризацией функции $f(x)$, имеющей неинтегрируемую особенность вида $\frac{1}{(x - x_0)^{\alpha}}$, называется функционал $(f, \varphi)$, который для пробных функций $\varphi(x)$, равных нулю в окресности точки $x_0$, выражается интегралом $\int f(x) \varphi(x) \, dx$.

\paragraph{Сингулярная обобщенная функция $\mathcal{P}{\frac{1}{x}}$. Формула Сохоцкого.} \footnote{Источник: Владимиров В.С. стр. 77}

В соответствии с определением регулярной обобщенной функции, сингулярную функцию нельзя отождествить ни с какой локально интегрируемой функцией. Простейшим примером такой функции является $\delta$-функция Дирака.

Введем линейный функционал $\mathcal{P}{\frac{1}{x}}$, действующий по формуле
\begin{equation*}
	(\mathcal{P}{\frac{1}{x}}, \varphi) = \underbrace{\text{V.p.} \int \frac{\varphi(x)}{x} \, dx}_{\text{a.k.a. инт. в смысле гл. зн-я}} = 
	\lim \limits_{\epsilon \to +0} \Big(\int_{-\infty}^{-\epsilon} + \int_{+\epsilon}^{+\infty}\Big) \frac{\varphi(x)}{x} \, dx, \quad \varphi \in \mathcal{D}(\mathbb{R}^1).
\end{equation*} 

Установим равенство
\begin{equation}
	\label{huinya_s_potolka}
	\lim \limits_{\epsilon \to +0} \int \frac{\varphi(x)}{x + i \epsilon} \, dx = - i \pi \varphi(0) + \text{V.p.} \int \frac{\varphi(x)}{x} \, dx, \quad \varphi \in \mathcal{D}.
\end{equation}

Действительно, если $\varphi(x) = 0$ при $\abs{x} > R$, то 
\begin{align*}
	\lim \limits_{\epsilon \to +0} \int \frac{\varphi(x)}{x + i \epsilon} \, dx = \lim \limits_{\epsilon \to +0} \int_{-R}^{R} \frac{x - i \epsilon}{x^2 + \epsilon^2} \varphi(x) \, dx = \\
	= \varphi(0) \lim \limits_{\epsilon \to +0} \int_{-R}^{R} \frac{x - i \epsilon}{x^2 + \epsilon^2} \, dx + \lim \limits_{\epsilon \to +0} \int_{-R}^{R} \frac{x - i \epsilon}{x^2 + \epsilon^2} [\varphi(x) - \varphi(0)] \, dx = \\
	= - 2 i \varphi(0) \lim \limits_{\epsilon \to +0} \arctan{\frac{R}{\epsilon}} + \int_{-R}^{R} \frac{\varphi(x) - \varphi(0)}{x} \, dx = \\
	= - i \pi \varphi(0) + \text{V.p.} \int \frac{\varphi(x)}{x} \, dx.
\end{align*}

Соотношение \eqref{huinya_s_potolka} означает, что существует предел последовательности $1/(x + i \epsilon)$ в $\mathcal{D}$, $\epsilon \to +0$, который мы обозначим $1/(x + i0)$, и этот предел равен $-i \pi \delta(x) + \mathcal{P}{\frac{1}{x}}$. Итак,
\begin{equation}
	\label{sohoc}
	\frac{1}{x + i 0} = - i \pi \delta(x) + \mathcal{P}{\frac{1}{x}}.
\end{equation}
Аналогично, 
\begin{equation}
	\label{sohoc'}
	\frac{1}{x - i0} = i \pi \delta(x) + \mathcal{P}{\frac{1}{x}}. 
\end{equation}

Формулы \eqref{sohoc} и \eqref{sohoc'} называются формулами Сохоцкого. Они широко используются, например, в квантовой физике.



